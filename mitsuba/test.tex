\documentclass{mitsuba}

\usepackage{kantlipsum}
\usepackage{layouts}

\title{Mitsuba test document}
\author{Marko V. Kant}

\begin{document}

\maketitle

\tableofcontents

\chapter{Introduction}

\kant[1]

\section{A section}

\kant[2-5]

\section{Another section}

\kant[6-8]

Animals, as the Chinese philosophers were well aware, can most conveniently
and efficaciously (not to mention with great auspicion) be divided into:
\begin{itemize}
\item those that belong to the Emperor,
\item embalmed ones,
\item those that are trained,
\item suckling pigs,
\item mermaids,
\item fabulous ones,
\item stray dogs,
\item those included in the present classification,
\item those that tremble as if they were mad,
\item innumerable ones,
\item those drawn with a very fine camelhair brush,
\item others,
\item those that have just broken a flower vase,
\item those that from a long way off look like flies.
\item \kant[36-38]
\end{itemize}

\kant[9-10]

\kant[22-35]

\chapter{Discussion}

\section{Point}

\kant[11-15]

\section{Counterpoint}

\kant[16]

\subsubsection{Foo}

\kant[17]

\subsubsection{Bar}

\kant[18]

\subsubsection{Baz}

\kant[19]

\subsubsection{Quux}

\kant[20]

\chapter{Conclusion}

\kant[21]

\clearpage

\begin{figure*}
\currentpage
\printheadingsfalse
\oddpagelayouttrue
\twocolumnlayouttrue
\pagedesign
\caption{Odd page layout}
\end{figure*}

\begin{figure*}
\currentlist
\printheadingsfalse
\listdesign
\caption{List layout}
\end{figure*}

\end{document}
