% $Id: fontanvil.tex 4451 2015-11-26 21:08:08Z mskala $

\documentclass{mitsuba}

\usepackage{framed}
\usepackage{makeidx}
\usepackage{tikz}

\makeindex

%%%%%%%%%%%%%%%%%%%%%%%%%%%%%%%%%%%%%%%%%%%%%%%%%%%%%%%%%%%%%%%%%%%%%%%%

\setlength{\FrameRule}{\fboxrule}
\setlength{\FrameSep}{0.5em}
\setlength{\OuterFrameSep}{0.5em}

\newcommand{\smhv}[1]{\textsf{\textbf{#1}}}

\newcommand{\PEFuncRef}[1]{%
\section*{#1}\label{func:#1}%
\index{#1@{\texttt{#1()}}|smhv}

}

\newcommand{\FFdiff}{\marginpar{\LARGE \raisebox{-1.0ex}{\textit{ff}}}}

%%%%%%%%%%%%%%%%%%%%%%%%%%%%%%%%%%%%%%%%%%%%%%%%%%%%%%%%%%%%%%%%%%%%%%%%

\title{FontAnvil 0.4}
\author{Matthew Skala}
\date{\today}

\begin{document}

\maketitle

%%%%%%%%%%%%%%%%%%%%%%%%%%%%%%%%%%%%%%%%%%%%%%%%%%%%%%%%%%%%%%%%%%%%%%%%

\begin{copyrightpage}
Visit the FontAnvil home page at
\url{http://tsukurimashou.osdn.jp/fontanvil.php}

\vspace*{1in}

FontAnvil user manual\\
Copyright \copyright\ 2014, 2015\quad Matthew Skala

\vspace{\baselineskip}

This document is free: you can redistribute it and/or modify
it under the terms of the GNU General Public License as published by
the Free Software Foundation, either version 3 of the License, or
(at your option) any later version.

\vspace{\baselineskip}

This document is distributed in the hope that it will be useful,
but WITHOUT ANY WARRANTY; without even the implied warranty of
MERCHANTABILITY or FITNESS FOR A PARTICULAR PURPOSE.  See the
GNU General Public License for more details.

\vspace{\baselineskip}

You should have received a copy of the GNU General Public License
along with this document.  If not, see \url{http://www.gnu.org/licenses/}.

\vspace*{0.75in}

\emph{The above license for the document itself notwithstanding,} the
FontAnvil software described in this document comprises the work of many
different copyright holders who have licensed their contributions under a
variety of terms.  The package as a whole is GPL, but some portions of it
are also available under less restrictive licenses.  See the ``Licensing''
chapter of this document for more information.

\vspace{\baselineskip}

Anvil clip-art by ``Gerald\_{}G,'' public domain.
\end{copyrightpage}

%%%%%%%%%%%%%%%%%%%%%%%%%%%%%%%%%%%%%%%%%%%%%%%%%%%%%%%%%%%%%%%%%%%%%%%%

\tableofcontents

%%%%%%%%%%%%%%%%%%%%%%%%%%%%%%%%%%%%%%%%%%%%%%%%%%%%%%%%%%%%%%%%%%%%%%%%

\input{intro.tex}
% $Id: building.tex 4482 2015-12-07 20:56:23Z mskala $

\chapter{Building FontAnvil}

There are no immediate plans to build binary distribution packages; to use
this code you will have to build it yourself from sources.

\section{From a version control checkout}

FontAnvil is available by anonymous Subversion checkout from
\url{http://svn.osdn.jp/svnroot/tsukurimashou/trunk/}. 
That is the main public repository for FontAnvil source code, and checking
out from there is (at least for the moment, while the code is in flux) the
preferred way to obtain FontAnvil.  The Tsukurimashou repository as a whole
is also mirrored on Github at
\url{https://github.com/mskala/Tsukurimashou.git}.

The version control system does not track some files that would be included
in a distribution tarball but can be built automatically from source files
already under version control.  That includes some parts of the build
system, such as ``configure,'' and several source files that are
automatically generated by Icemap.  \emph{You must build Icemap
before you can build FontAnvil from a version control checkout.} If you have
checked out the entire project, then Icemap can be found in
the \texttt{icemap/} subdirectory, sibling to \texttt{fontanvil/}.

To go this route you will also need to have all the dependencies of Icemap
installed, which include but are not limited to BuDDy (available at
\url{https://sourceforge.net/projects/buddy/}) and PCRE (available at
\url{http://www.pcre.org}).  It should go without saying that these must be
installed in such a way that they \emph{actually work}.  I've had reports of
Arch Linux, in particular, using a nonstandard dynamic linker configuration
such that installing these libraries in the normal way will allow software
using them (such as Icemap) to compile but not run.  Icemap's configure
script will attempt to test for this condition and warn you, but it is not
foolproof.

If you wish to build from a version control checkout, it is probably easiest
to check out and configure the entire Tsukurimashou Project even if you are
primarily interested in FontAnvil.  These instructions assume you have done
that.  It is also assumed you have a complete installation of the standard
GNU tools for building C programs, including the gcc compiler, Autoconf,
Automake, flex, and so on.  Many of these would not be needed for building
from a distribution package.

From the root of the Tsukurimashou checkout, run:
\begin{verbatim}
autoreconf -i
\end{verbatim}

This will create configure scripts and other infrastructure for all packages
in the project.  The \texttt{-i} option tells \texttt{autoreconf} to
automatically create any missing scripts (notably, the one named
\texttt{missing}) that it thinks it needs.

The \texttt{autoreconf} command
is likely to produce many error and warning messages, but it should work
nonetheless if the Autotools versions are suitable.  In particular, note
that when \texttt{autoreconf} runs it may generate repeated ``underquoted
definition'' warnings in relation to files in /usr/share/aclocal or similar. 
\emph{This is the fault of some other package on your system;} it is not
related to FontAnvil and should be harmless to the FontAnvil build.  The
issue is that many packages (IMLIB is one common culprit) install M4 files
globally to provide customized Autoconf tests, and then if those files are
buggy, \texttt{autoreconf} complains whenever it is invoked even if, as
here, the buggy files are not actually being used.

The top-level Tsukurimashou build will automatically take care of building
Icemap and FontAnvil in order to build Tsukurimashou, but assuming you don't
want to build the entire project that way, the next step is to build Icemap:

\begin{verbatim}
cd icemap
./configure
make
\end{verbatim}

The configure script supports \texttt{-{}-help} and most of the usual
options; run that and make appropriate modifications if you wish to
customize the Icemap build.  Do not ignore error messages from configure.

After building Icemap, you can build FontAnvil.  The FontAnvil build will
automatically detect Icemap if it is in the sibling directory created by a
source control checkout; it should not be necessary to install Icemap
elsewhere first.  However, running Icemap requires several files of
character-database information from the Unicode Consortium.  The
\texttt{make download} target will automatically go on the Net using
\texttt{wget} to download them; alternatively, you could read the Makefile
to find which files are needed and install them some other way.  Thus, the
simplest sequence of commands is:

\begin{verbatim}
cd ../fontanvil
./configure
make download
make
\end{verbatim}

At this point FontAnvil should be ready to install with \texttt{make
install} (root privilege may be necessary for that), or to test with
\texttt{make check} (test suite is likely to fail at this level of
development progress).

\section{From a distribution package}

Distribution packages are available from the FontAnvil home page at
\url{http://tsukurimashou.osdn.jp/fontanvil.php}.  Be aware that these
packages will \emph{usually} be out of date; and as a consequence of
Murphy's Law, it is usual for me to discover many critical bugs immediately
after releasing a package.

Building from a package is much the same as building from a version
control checkout, minus the need to run Autotools and Icemap.  Most of the
automatically generated source files are included in the package, so it is
not necessary to have Unicode character set files, Icemap, flex, and so on.
Unpack the tarball and do the usual Autotools build:

\begin{verbatim}
./configure
make
# as root:
make install
\end{verbatim}

\section{Special notes for Windows users}

I cannot realistically offer support for any platforms except Linux and
MacOS, which are the ones for which I have access to testing machines. 
However, FontAnvil's build system is intended to be as portable as
realistically possible, and as of shortly before version 0.4, Jeremy Tan has
reported some success at compiling FontAnvil under Windows.  He has posted
unofficial Windows binaries of a patched 0.3 version at
\url{http://sourceforge.net/projects/fontforgebuilds/files/fontanvil/} and
indicated some willingness to possibly continue updating them.  I will also
fold back into the mainline any changes that are reasonably unintrusive and
help with building on Windows.

Spaces in pathnames, such as ``\texttt{C:\textbackslash{}Program
Files},'' are a serious problem for all Autotools-based build systems. 
FontAnvil \emph{probably will not} be able to build if you locate the source
code in a path with a space in it.  It may have \emph{some}, but not
\emph{complete}, ability to handle tools such as compilers and \LaTeX\ that
may be located in pathnames with spaces.  The \texttt{configure} scripts
will attempt to detect and warn you about such paths, but be aware that some
may cause problems even if \texttt{configure} produces no warnings, and the
build may be able to compensate even if \texttt{configure} does produce
warnings.

All I can recommend is to try it and then attempt to work around
any problems that occur, by moving things that break into paths that do
not include spaces.  I'd also like to emphasize that this is a problem
with \emph{all} Autotools build systems, not only FontAnvil's.  If you want
to make a general practice of building Unix software on Windows, the only
course of action that will really work is to eliminate filename spaces from
your environment.  You cannot expect every software package to work around
Windows's wacky filenames, and you cannot expect FontAnvil to do so.

After FontAnvil is built there should be no problem with using it in an
environment that has spaces in pathnames; this is only an issue for the
build.

\section{Testing}

FontAnvil includes a test suite, available by running \texttt{make check}. 
Most of the tests are inherited from FontForge, and the test suite had been
unmaintained and unused in FontForge for several years before FontAnvil
picked it up.  As a result, it does not have good coverage, and some of the
tests require files that I cannot legally ship.  In some cases I have not
even been able to identify what the missing files are.  If you run the test
suite, you should expect many of the tests to be skipped for missing
necessary files, and some of them to fail; and even if by some chance you
managed to get the entire suite to pass, it is unlikely that that would
really mean the software was working properly.  Having a better test suite
and software that passes it are long-term goals for the project.

After running the tests, you can read the file \texttt{tests/coverage.log}
to see a summary of how many of the PE script built-in functions were
covered by the test suite.  As of this writing, it is about 15\%.

The option \texttt{--enable-valgrind} to \texttt{configure} will make the
test suite run inside Valgrind, if you have that installed.  Valgrind
makes the tests much slower, and (with the latest versions as of this
writing) it causes at least one of them to fail on Mac OS X through no fault
of FontAnvil's, because of Valgrind's lack of support for features used by
the Mac system libraries.  For these reasons it is not the default. 
However, if enabled it will produce a lot more information in the log files
about the many ways in which FontAnvil abuses memory, and that may be useful
in debugging.

The fact that the test suite fails prevents \texttt{make distcheck} from
working.  If you set the environment variable \texttt{distcheck\_hack} to
\texttt{0.4} (or the new version number, if I continue using this hack in
future versions), then the tests I \emph{expect} to fail will be disabled,
so that \texttt{make distcheck} can be tricked into accepting the package
for distribution.  In the long run, this is probably a Bad Thing.

\section{FontAnvil and Tsukurimashou}

FontAnvil's reason for existence is to support Tsukurimashou, and its source
control repository is a subdirectory of the Tsukurimashou source control
repository.  FontAnvil does not require Tsukurimashou to build. 
Tsukurimashou will look for FontAnvil and use it if found, whether installed
on the system or inside a subdirectory of its own build.  The Tsukurimashou
top-level build will also automatically build Icemap and FontAnvil, as
needed, if started on a system without them and you don't override this
behaviour.  Note that this represents a change from earlier version of
Tsukurimashou, which required FontAnvil to be built and installed separately
first.

All bug reports and other tickets for FontAnvil should be filed through
the Tsukurimashou ticket tracker at
\url{http://osdn.jp/projects/tsukurimashou/ticket/}.  Set the
``Component'' field to ``FontAnvil.''

As a courtesy to Github users, Tsukurimashou's entire source control system
(including FontAnvil) is mirrored in my Github account at
\url{https://github.com/mskala}.  But osdn.jp remains the
authoritative public home of the project.  You are welcome to clone the
repository---that is why it's there---but the semi-automated gateway from
Subversion to Git is one-way.  Do not file tickets for FontAnvil on Github.

Do not file tickets for FontAnvil in the FontForge tracker on Github!  They
are completely separate pieces of software with different maintainers and
different goals, notwithstanding the historically shared code and
notwithstanding that the FontForge people have graciously given me developer
status on their project.

\clearpage

\chapter{Running FontAnvil}

FontAnvil is a script interpreter, so in normal operation it is assumed you
already have a script file for it to interpret.  Scripts are written in the
PE script language described elsewhere in this document.  Script files for
FontAnvil are traditionally given the filename extension \texttt{.pe} (for
``PfaEdit''); the extension \texttt{.ff} is also popular.  Invoking the
FontAnvil interpreter then proceeds on more or less the same lines as
invoking any other script interpreter.

\section{Command-line options}

\begin{framed}
FontAnvil's command-line syntax attempts to achieve some degree of FontForge
compatibility.  However, as of March~2014, FontForge contains at least six
different command-line
parsers,\footnotemark{} and
also sometimes hands its command lines off to Python for parsing, so that
options interact in complicated ways with each other, with compile-time
settings, with operating system shebang support and whether stdin is a
terminal or pipe, and so on.  FontAnvil does not attempt to match all of
this behaviour exactly.
\end{framed}
\footnotetext{https://github.com/fontforge/fontforge/issues/1277}

FontAnvil uses GNU \texttt{getopt\_long\_only} to parse command-line
arguments; this has the consequence that long options may be specified with
\texttt{-} (one hyphen) or \texttt{-{}-} (two hyphens) as the flag sequence;
using \texttt{-{}-} is the modern-day Unix convention, but \texttt{-} may be
preferable for FontForge compatibility.  Short options require a single
hyphen.  Options recognized are as follows.

\begin{description}
  \item[\texttt{-command} $\langle cmd\rangle$,
    \texttt{-c} $\langle cmd\rangle$]
  Execute a PE script command given literally on the command line.  FontAnvil
  will \emph{not} look for a script file name on the command line if this
  option is specified; all arguments starting with the first non-option
  argument become arguments to the script.  Only the last invocation of this
  option will be used; unlike, for instance, Perl, it is not possible to
  build up a multi-line script by specifying \texttt{-c} multiple times.

  \item[\texttt{-dry}, \texttt{-d}]
  Activate a poorly-documented ``dry run mode'' built into some parts of
  the FontForge PE script interpreter.  This appears to
  be intended for syntax checking.  \emph{Most}, but not necessarily
  \emph{all}, commands will be skipped.  \emph{I do not promise that new
  code added in FontAnvil will necessarily respect this mode.}

  \item[\texttt{-help}, \texttt{-usage}, \texttt{-h}]
  Display a command-line option help message, and terminate without
  executing a script.

  \item[\texttt{-lang} $\langle cmd\rangle$,
    \texttt{-l} $\langle cmd\rangle$]
  Specify interpreter language.  If this option is given with the value
  ``\texttt{ff}'' then it will be ignored for compatibility.  Any other
  value is a fatal error.

  \item[\texttt{-quiet}, \texttt{-q}]
  Reduce the verbosity of error, warning, and informational messages sent to
  standard error.  This option may be specified more than once to reduce the
  messages even further.  Note it may not function exactly like the
  similarly-named option in FontForge.  FontForge often ignores the
  command-line option and controls message verbosity in some modules by
  other means including compile-time options, while FontAnvil attempts to
  unify all verbosity control under this one option.

  \item[\texttt{-verbose}, \texttt{-V}]
  Increase the verbosity of error, warning, and informational messages sent
  to standard error.  This option may be specified more than once to
  increase the messages even further.

  \item[\texttt{-nosplash}, \texttt{-script}, \texttt{-i}]
  Ignored for compatibility.

  \item[\texttt{-version}, \texttt{-v}]
  Display a version and copyright banner, and terminate without executing a
  script.

  \item[\texttt{-{}-}]
  Terminate option scanning.  All subsequent arguments will be treated as
  ``non-option arguments'' (thus eligible to become script file names or
  script arguments) even if they resemble FontAnvil options.  This would be
  what you might use if for some reason you needed to execute a script file
  that was named exactly ``\texttt{-script}.''
\end{description}

The first non-option command line argument will be taken as the filename of
a script file to execute, unless the \texttt{-c} option or one of its
synonyms has overridden this behaviour.  If the filename so specified is a
single hyphen, or if there are no non-option command line arguments
at all, then FontAnvil will enter \emph{interactive mode}, reading
commands from standard input, as described later in this chapter.  Any
command line arguments after any script filename will be passed
into the script in the variables \texttt{\$1},
\texttt{\$2}, and so on---even in the cases of \texttt{-c} and interactive
mode.

\begin{framed}
Option scanning stops at the first non-option argument encountered, which
will usually be treated as the script filename.  Any arguments after that
become arguments to the script (passed in the variables \texttt{\$1},
\texttt{\$2}, etc.) and not options for the FontAnvil interpreter.  For this
reason, options \emph{must} precede the script file name on the FontAnvil
command line.  For maximum compatibility, the \texttt{-script} option should
be the last option if you use it at all, with the script file name in a
separate argument, not attached using \texttt{=}.
\end{framed}

\section{Shebang}

FontAnvil may be invoked using the shebang convention.  Place a line
something like ``\texttt{\#!/usr/local/bin/fontanvil}'' at the top of a
file, and make the file executable, to create a script that can be run like
any other program and will automatically use FontAnvil as the interpreter.

Details of shebang support vary depending on the operating system.  On most
systems, the shebang line must specify an absolute path, and the
\texttt{env} program may be used to search for a command name in the path to
avoid hardcoding the absolute location of the interpreter into a script. 
There are also special considerations applicable to the length of the
interpreter path, arguments specified in the shebang line, and so on. 

\emph{FontAnvil does not have any special support for shebang.}  In
particular, it does not scan the script to look for its own name in the
shebang line.  Since the shebang line by definition starts with the comment
character \texttt{\#}, it will be skipped as a comment.  FontAnvil just
takes the script file name as an argument from the operating system, and
(assuming the script name does not happen to be something weird that looks
like an option) executes it, with any remaining arguments becoming arguments
to the script.  This is normally the desired behaviour.  However, be
\FFdiff aware
that it is a technical difference from FontForge, which attempts to
determine whether it was invoked via the shebang mechanism and do smart
things depending the answer, including working around operating systems that
support this feature only poorly.  FontForge may possibly \emph{require}
options in the shebang line in at least some cases, to select which
scripting language it will use.

If you try to specify command-line options in the shebang line, then
depending on your operating system's support it is possible that FontAnvil
will not see the options even though FontForge would.  Some operating
systems have unintuitive behaviour regarding options specified in the
shebang line; for instance, combining all options into a single string
passed as one argument instead of splitting them on spaces.  For this
reason, authorities on Unix often recommend against using options in the
shebang line at all; nonetheless, people continue doing it.

For maximum compatibility with both interpreters, I suggest writing shebang
lines in PE script files as you would write them for FontForge (including
mentioning the filename ``\texttt{fontforge}''), and then invoking FontAnvil
on the files by other means when desired.  That way, FontForge will see the
interpreter name and any options it wants, and FontAnvil will ignore them.

\section{Interactive mode and readline}

FontAnvil is intended primarily for non-interactive use.  However, if it is
invoked without a script file name, or with ``\texttt{-}'' (a single hyphen)
as the script file name, then it will enter a special \emph{interactive
mode}, where it reads commands from standard input and executes them
immediately, line by line, rather than reading from a script file.  This can
be convenient for one-off editing tasks and testing the syntax and behaviour
of script commands.

If FontAnvil was compiled with the GNU Readline library and detects that
standard input is a terminal, then interactive mode will also offer
command-line editing and history using Readline.  The usual Readline
keystrokes (such as up and down arrows to recall earlier-typed command
lines) become available in this mode, and there are some minor changes to
the output formatting (in particular, the display of a command prompt) to
make it friendlier for interactive users.

\chapter{Data model}
\label{sec:data-model}

Very many difficulties users have with font editing (both scripted and
interactive) come from an incomplete understanding of the data model
involved: what entities exist in a font and a font editor and what
relationships those entities have with each other.  The distinction between
glyphs and characters seems to be an especially frequent cause of confusion. 
This chapter attempts to describe FontAnvil's data model in a way that will
be useful to script programmers.

\section{Fonts in memory}

Because PE script was originally designed for controlling a GUI font
editor, it treats fonts as documents to be opened and closed, much as
a GUI editor might.

At any given time there exists a global set of fonts that are \emph{open}. 
These are stored in RAM.  Open fonts are associated with filenames, even if
the filenames do not actually exist on disk; the interpreter will assign
temporary filenames (usually similar to ``\texttt{Untitled1.sfd}'') to fonts
that were created in memory and not loaded from files.  The filenames should
be unique (no more than one open font sharing a filename), and it's not easy
to create a situation where they are non-unique, but I suspect that having
distinct open fonts with duplicate filenames may be technically possible and
likely to trigger bugs if attempted.

The set of open fonts is technically a sequence (with a specific order), not
an unordered set, and the order is visible to scripts through the
\verb|$firstfont| and \verb|$nextfont| built-in variables, but this fact is
seldom important.

At most one of the open fonts may be the \emph{current font}.  This state is
global.  Most font-editing operations implicitly apply to the current font. 
The \texttt{Open()} built-in function sets the current font, but its exact
behaviour is context-sensitive.  If the specified filename is already an
open font, then \texttt{Open()} just sets the current font to that one.  If
the specified filename is \emph{not} already an open font, then
\texttt{Open()} loads it from disk, causing it to become an open font, before
setting the current font to that one.

When the interepreter starts up, the set of open fonts is empty and there is
no current font.  In this state, any operation that implicitly refers to the
current font will fail; scripts can only use a small subset of the language
to create or open a font and make it current.  The \texttt{Close()} built-in
function removes the current font from the set of open fonts (without saving
it to disk---that must be done as a separate operation if saving is desired)
and places the interpreter back into the state of having no current font. 

The \FFdiff existence of a no-font state is a difference between PE script
interpreters (both FontAnvil and FontForge when run as a command-line
interpreter) and the FontForge GUI.  The GUI insists on always having at
least one open font and always having a current font, enforcing this rule by
automatically loading fonts from earlier editing sessions, automatically
creating a new empty font if the load fails, choosing another open font to
be current when one is closed, and terminating the program when the last
font is closed.

\section{Glyphs and slots}

Here are two pictures of Don Quixote.\footnote{Left: Gustave
Dor\'{e}, 1863.  Right: Honor\'{e} Daumier, 1868.}

\begin{center}
\includegraphics[width=1.3in]{quixote-dore.jpg}\qquad
\includegraphics[width=1.3in]{quixote-daumier.jpg}
\end{center}

The pictures are different, but they are pictures of the same fictional
character.  In the same way, we can have several pictures of the same
character in a writing system.

\begin{center}
\scalebox{4}{a}\qquad
\scalebox{4}{\textit{a}}\qquad
\scalebox{4}{\texttt{a}}
\end{center}

What these three ``a''s share is the \emph{character}; what they do not
share is the \emph{glyph}.  Fonts contain collections of glyphs, concrete
things, which are pictures of characters, abstract things.  It is often the
case that in any given font, each glyph corresponds to exactly one character
and vice versa; but there are many important exceptions to that rule.  To
understand how FontAnvil processes fonts it is important to bear in mind
that fonts are collections of glyphs, and glyphs are not the same thing as
characters despite being closely connected with characters.

This glyph (the classic ff-ligature) is not associated with one character,
but with a sequence of two characters:

\begin{center}
\scalebox{5}{ff}
\end{center}

There is no double-f character, as a separate abstract entity distinct from
just two ordinary ``f''s in a row, in the English language.  There is no
standard character code for such a thing.\footnote{Actually, there is one in
Unicode, but you're not supposed to really use it; the details of that are
beyond the scope of this discussion.} Nonetheless a font for high-quality
typesetting of English must contain a glyph for this double-f entity
that is not quite a character.  Despite such exceptions, one glyph per
character is true most of the time in English.  In some other languages, the
conceit of glyph-character equivalence breaks down entirely.  In Arabic, for
example, many characters have four or more visual forms requiring
separate glyphs in a font, depending on how they connect to neighbouring
characters.

FontAnvil represents a font in memory as including a zero-based array of
\emph{glyph slots}.  The array is variable-sized, but it is a true array
data structure, not a list: all slots from index 0 up to one less than the
number of slots exist as long as the in-memory font does.  Creating a new
glyph means overwriting the possibly-blank previous contents of some slot. 
Destroying a glyph means filling the slot with a blank glyph, but the slot
continues to exist.  Changing the number of slots can only be done by
increasing or decreasing the length of the array, and encodings (described
in the next section) may constrain the number of slots.

Glyph slots in a font always have \emph{glyph numbers} which are their
indices in the array.  Glyphs in a font cannot meaningfully be said to be in
any specific order other than the order determined by the array indices. 
Every glyph has a number, and every number has a glyph slot.  No two glyphs
can have the same slot; two slots may contain identical glyphs, or even a
``reference'' from one to the other, but the two slots' contents will not
truly be the same entity.  A glyph slot might be \emph{blank}, that is
devoid of outlines and other data, and people usually think of such slots as
not really being glyphs.  But some attributes of a glyph slot (such as its
name) are required to always have non-null values, even on blank slots.

\begin{framed}
FontAnvil's in-memory glyph slots can be blank but never truly empty. 
However, most on-disk file formats \emph{do} have a concept of a glyph
failing to exist at all.  Blank slots are usually not written when saving
from memory to disk, and loading a file from disk to memory that does not
fill all slots will usually leave the others blank.
\end{framed}

Every open font has a \emph{selection}, which specifies a set of the glyph
slots (more about glyph slots in the next section).  Many editing operations
implicitly operate on the selection of the current font.  Although every
open font has a selection, usually only the selection of the current font is
relevant.  Open fonts retain their selections through changes in which open
font is current.

The selection is actually a sequence, not a set, of glyph slots.  That means
the slots in it can be selected in a specific order.  This distinction is
relevant in the FontForge GUI, where you can select several glyphs in a
specific order, open a ``Metrics'' window, and have them come up in the
order you selected them.  However, each glyph slot can appear at most once
in the selection, and the order of the selection is seldom if ever
observable or controllable from the scripting language.  It is usually
better to think of it as a set with no specific order, not as a sequence.

The selection is applied at the level of glyph slots.  It is perfectly
possible for the selection to include blank glyph slots, because it is
defined as a set of slots, not a set of non-blank glyphs.  Nonetheless, one
often only cares about the non-blank glyphs, and some commands for
manipulating the selection will automatically limit themselves to slots that
are not blank.

Glyph slot numbers \emph{may or may not} be closely connected to Unicode,
ASCII, or other character codes, depending on issues discussed in the next
section.  It is important to be aware that glyph slot numbers are not the
same type of entity as character codes, despite sometimes having equal
numerical values.

\section{Encodings}

Fonts do not exist solely to be edited with FontAnvil.  To be useful, a font
must eventually be saved in some format understandable by a word processor
or similar application; and the external software must be able to associate
the glyphs in the font with the characters in text that will be typeset
using the font.  There will necessarily exist some \emph{code} that
associates numbers called \emph{code points} with the different characters
that might appear in text, and a font file must explicitly or implicitly
describe which code points go with which glyphs.  It is worth emphasizing
that code points refer to characters, \emph{not glyphs}.

The ASCII code is familiar to many computer users, especially in the
English-speaking world, but is inadequate for languages other than English. 
Today, nearly everybody uses a code called Unicode.  Unicode's code points
coincide with ASCII for all the characters covered by ASCII; but it also
covers many thousands of other characters.  It is supposed to be a universal
code for all text in all human languages.  But Unicode did not always exist and its use was not always
universal.  Most common font formats are designed to also accomodate
character encoding schemes predating Unicode or simply other than Unicode. 
Frequently, a font file will include translation mappings for several
different codes, in the hope that software using the font can find its own
preferred code among the choices.  There needs to be a way to translate code
points (which identify characters) into glyph numbers (which identify
glyphs), even in cases like ligatures and variant glyphs where this
translation is more complicated than a simple one-to-one lookup.

FontAnvil includes several mechanisms for addressing these issues, but the
most significant is that of the \emph{encoding}.  The encoding is a
property of each font (global to the font) and describes a set of
assumptions and constraints about the relationships between code points and
glyph numbers.

\begin{framed}
Every glyph slot in a font always has a glyph number, no two slots can have
the same number, and the set of glyph numbers that exist is always the set
of integers from 0 up to one less than the number of glyphs in the font. 
Every code point designates at most one glyph slot.  But a glyph slot might
have zero, one, or more than one code point; a code point might have no
glyph slot; and the set of code points that exist might not be a simple
interval of the integers.  Nonetheless, the font's encoding may trigger the
enforcement of constraints that make the code point situation less
complicated.
\end{framed}

Most of the possible values of the encoding field are associated with
standardized character codes defined by external organizations.  Each code
defines a meaning for a range of code points from 0 up to some number that
depends on the code.  \emph{When the encoding is associated with an
externally standardized code other than Unicode}, FontAnvil enforces the
following constraints:
\begin{itemize}
\item There must be at least as many glyph slots as there are code points in
the code.
\item Glyph slots in the range 0 through one less than the number of code
points in the code correspond one-to-one with the code points.
\item Any glyph slots with greater indices have no code points.
\end{itemize}

The case of \emph{unencoded glyphs}, those in glyph slots beyond the end of
the code point range specified by the encoding, is important.  Glyphs like
the ff-ligature mentioned earlier, or the alternate forms of letters in a
script like Arabic, usually fall into this category.  When a word processor
typesets text using a font, it starts out by translating the code points
into glyphs one-for-one according to the basic code points of the glyph. 
The result of that translation cannot contain unencoded glyphs.  But the
basic one-for-one translation of code points to glyphs is only a starting
point.  Further processes can merge and split glyphs so that more than one
character can be typeset with one glyph, one character can be typeset with
more than one glyph, and which glyph goes with which character can be
different in different contexts.  These further processes can bring
unencoded glyphs into play.  The encoding does not specify the number of
unencoded glyph slots that may exist after the range of encoded glyphs.  The
unencoded glyph slots may be manipulated by built-in functions like
\texttt{SetCharCnt()}; encoded glyph slots may not be added or removed.

Each glyph slot has a property called the \emph{Unicode number}.  This is a
code point (in the code that is named Unicode), but I am going to call the
number in this field just a ``number'' to distinguish it from the code
points that exist in non-Unicode encodings.  When the encoding is one of the
standardized non-Unicode encodings, the constraint is enforced that the
Unicode number must be the correct translation (using the iconv library) of
the glyph number for encoded glyphs, or the null value of -1 for unencoded
glyphs.  For example, if the font's encoding is KOI8-R (commonly used for
Russian text), then glyph slot number 241 is for the letter ``ya,'' which
looks like a backwards R.  FontAnvil will enforce the constraint that this
slot's Unicode number is 0x042F, which is the Unicode code point for that
letter.  ``The constraint is enforced'' means that if you try to change the
value of the Unicode number field, the font's encoding will be immediately
changed to Custom.  Editing the Unicode numbers is not compatible with
keeping the encoding and its fixed mapping.

But not every value for the font's global encoding field is associated with
an external standard other than Unicode.  When the font's encoding field
refers to some form of Unicode, or does not refer to an external standard,
then additional special considerations apply; and in fact, these special
cases are the most popular and useful values for the encoding field.

When the encoding is set to Custom, few encoding-related constraints are
enforced.  There may be any number of glyph slots.  Any slot may have a
Unicode number, or not, and there is not necessarily any relationship
between the Unicode numbers and the glyph slots.

There are two Unicode encoding options, Unicode (BMP) and Unicode (Full). 
These each behave more or less like the non-Unicode standardized encodings. 
One difference is that it appears sometimes possible to set the Unicode
number of a glyph slot such that it does not match its glyph number.  This
may be a bug.  FIXME investigate further - this description may possibly be
simplified if Unicode and non-Unicode turn out to really behave the same.

FIXME investigate and document the ``Original'' encoding option.

Glyph slots have names.  All glyph slots have non-empty names, including
blank slots, and all names must be unique within a font.  The names are
sometimes automatically assigned and may also be manipulated by script
commands; but constraints (including the requirement for uniqueness) will be
enforced on such manipulation.

\begin{framed}
People who think they want to edit glyph slot names are often wrong.
\end{framed}

FIXME document name lists

\section{The \texttt{.notdef} glyph}

FIXME

\section{The clipboard}

There is a global entity called the \emph{clipboard}, which holds glyph data
of the kind that might be stored in glyph slots, such as outlines, anchors,
and references.  The clipboard is like a font in that it can store a bunch
of slots' worth of data, in a definite order, but the clipboard is unlike a
font in that the slots do not have meaningful numbers and it does not store
slot attributes other than glyph data, such as slot names and code points. 

The usual way of using the clipboard is somewhat like using the clipboard in
any common GUI document editor: select some slots, do a cut or copy
operation, select some other slots (even in a different font), and do a
paste operation.  Here is typical code to copy the uppercase ASCII alphabet
from an existing font into a new font (leaving many things in the new font
empty or default, which may cause problems later):

\begin{verbatim}
Open("font1.sfd");
Select('A','Z');
Copy();
New();
Select('A','Z');
Paste();
Save("font2.sfd");
\end{verbatim}

One thing to be aware of is that \texttt{Paste()} always writes into the
selection, and so you must create a nonempty selection for \texttt{Paste()}
to be meaningful.  This differs from a word processor that can ``insert''
text; FontAnvil treats a font as fixed framework of glyph slots that can
only be changed by overwriting.  Inserting or deleting in the middle, in a
way that changes the number of slots that exist, would disrupt the framework
of the encoding and is rarely, if ever, what you really want.

\begin{framed}
A glyph slot's name is associated with the glyph slot, not with the glyph
data stored in the slot.  The slot name will not move with the glyph data
when the glyph data is cut and pasted into a new slot.  Unicode code points,
and any other encoding numbers, are also parts of the glyph slot and will
not move with cut and pasted glyph data.
\end{framed}

\section{Look-up tables}

FIXME

\section{Syntax, semantics, round trips, and reproducibility}

It is a very common complaint from users of font editors that someone loaded
a file, saved it without making any other changes, and ``the result was not
the same.''  Such complaints are often filed as bugs in the FontForge bug
tracker.  Dealing with them is not so easy as it might appear on the
surface, because what it means for two files to be ``the same'' is not so
simple as whether they contain the same sequence of bytes.  This section is
intended to clarify some of the relevant issues as they relate to FontAnvil,
and let users know what are and are not reasonable expectations about the
system's behaviour.

Consider the two ASCII strings ``\texttt{1.5}'' and
``\texttt{1.50000000000}''; are those the same?  As strings, they are not. 
But if these strings are representations of a real number, they represent
the same real number.  So if an editor loads one and saves the other, the
bytes in the file will have changed but the number will not have changed. 
We can say that the \emph{syntax} has changed but the \emph{semantics} have
not.  These are terms borrowed from the field of linguistics, where they
describe different levels of abstraction in the study of a language: syntax
refers to questions like what sequences of words are grammatically valid
sentences, while semantics refers to the meanings of words.

Note that if ASCII strings represent floating point numbers in low machine
precision, we might even have ``\texttt{1.50000000000}'' and
``\texttt{1.49999999999}'' referring to the same number.  The ``number''
that matters is the machine-precision binary number; two strings that both
result in the same machine-precision binary number are two strings with the
same meaning even if a human translating them instead to exact real numbers
would see a difference.

When FontAnvil, or any other font editor, loads a file from disk, the file
is converted into an in-memory representation that is designed to store the
semantics of the font.  When it saves the file, that in-memory
representation is converted into the disk file format.  As a result of this
process, in general it is reasonable to expect that loading and saving a
file will preserve the semantics of the data in the file, but in general
\emph{it is not reasonable to expect aspects of syntax beyond semantics to
be preserved.}

The result of a round-trip through any editor will in general be a file with
the same semantics, but not necessarily the same syntax.  In general that
statement is true of all software that edits complicated file formats,
including word processors and image editors as well as all font editors, not
only FontAnvil.  For reasons unknown to me, users of font editors seem
especially inclined to be surprised by the loss of non-semantic information
and to report it as a claimed ``bug,'' even though this is the normal,
expected, and pretty much inevitable behaviour of most computer
software.

Further confusion may ensue when a file is round-tripped through more than
one on-disk format, because different formats may define different
semantics, sometimes at the ontological level; that is, different formats
may be based on different models of what are the important concepts in the
problem domain.  For instance, some font formats make heavy use of
``references,'' which allow part of one glyph to automatically incorporate
another glyph.  Other font formats just do not contain references at all. 
Some font formats use quadratic splines to represent curves, while others
use cubic splines, and each of these is only approximately capable of
representing data from the other; a loss of precision necessarily occurs
when converting between them.  In general, information from one
format may need to be changed in irreversible semantic ways or even
discarded in order to convert a font to another format.  Converting back and
getting even the same semantics, let alone the same syntax, is not always a
reasonable expectation.  This issue is somewhat analogous to what occurs
when translating text successively through two or more very different human
languages.

The FontAnvil in-memory format attempts to be able to represent all the
semantic information in all the on-disk formats that FontAnvil supports; but
it still represents semantics only.  Since FontAnvil descends from FontForge
which descends from PfaEdit which was originally an editor for Postscript
only, the FontAnvil in-memory format may have some bias toward a Postscript
data model.  It nonetheless is intended to cover the semantics of all the
formats FontAnvil supports.

The FontAnvil SFD ``native'' format is a special on-disk format intended to
represent the semantics of the in-memory format as accurately as possible;
so, by design, it should be possible to round-trip between a font in memory
and a font on disk in SFD format, with no loss of semantics.  However, it is
still possible to find syntactic distinctions in SFD files that will not be
preserved by loading and saving.  For instance, one can mess around with
external software to re-arrange the sequence of glyphs in an on-disk SFD
file, only to have one's tampering automatically undone the next time the
file is loaded and saved because the order of entries in the file is not
semantic and all that really counts is the glyph numbers stored in the
appropriate fields.

From this general picture there emerge several points that describe how
FontAnvil relates to the syntax and semantics of font files.

\begin{itemize}
\item FontAnvil attempts to preserve semantics of font files through
a round trip of loading from and saving to any one supported foreign format. 
\item FontAnvil attempts to preserve semantics of font files through 
a round trip of loading a foreign format, saving and loading the data in SFD
format, and then saving back to the same foreign format.
\item When loading from one foreign format and saving to another, FontAnvil
attempts to preserve whatever semantics make sense in both formats; but a
round trip from one foreign format to another and back may not be lossless.
\item FontAnvil does not attempt to preserve syntax of any file format
beyond the semantics of the font inside, and in general it should be
expected that it usually \emph{will} change syntax.
\item FontAnvil does not attempt to preserve SFD file semantics specific to
features FontForge includes and FontAnvil does not; for instance, pickled
Python objects.
\item FontAnvil does not support SFD-like files written or
edited by software other than FontAnvil itself, except versions of FontForge
close to the Git repository version current as of March 2014.
\item FontAnvil does not support FontForge SFDir (SFD split into a directory
instead of a single file) format.
\end{itemize}

The Debian Project has recently embarked on a misguided project to demand
``reproducibility'' of everything in their world, which they define as
having the ability to compile software many times under different
circumstances and get byte for byte syntactically identical output.  I call
this \emph{extreme reproducibility} to allow distinguishing it from weaker
properties that might also be called reproducibility, and to emphasize what
an outlandish and cumbersome form this one actually is.  Several purported
reasons for extreme reproducibility to be desirable are put forward by the
project.  The claim that extreme reproducibility would be nice for security
is true, because multiple parties can independently compile something, check
that they got the same result, and thus be sure that they are all either
uncompromised or compromised in the same way.  But not everything that would
be nice to have is worth what it costs.

Absent rigorous audits of everything else that can go wrong, an extreme
reproducibility demand is far from a guarantee of security.  Demanding
extreme reproducibility precludes embedding metadata about the circumstances
of a build in the built file; as soon as there's so much as a timestamp
inside a file, byte for byte comparisons break.  Metadata is especially
important and useful in fonts and typesetting; and third-party file formats
in this application domain often have specific requirements for timestamps
and even per-save random ID numbers to be included in the files, making it
difficult to obey the file format definitions while also having extreme
reproducibility.

Randomized algorithms, and computations in which multiple threads work
simultaneously, may often generate differing but in some sense close enough
results when run more than once; anything of that nature is forbidden by
extreme reproducibility.  Cryptographic signature algorithms often require
the opposite of extreme reproducibility, with one-time nonce values that (on
pain of destroying the security guarantees) must \emph{never} be the same
from one run to the next; any part of a build process that involves such a
signature must be carefully segregated from any parts that are attempting to
guarantee extreme reproducibility.

With all that in mind, FontAnvil rejects the extreme form of
reproducibility, but embraces the weaker common-sense reproducibility
principle that running the same process more than once should produce
functionally identical results.

\begin{itemize}
\item FontAnvil scripts are intended to be \emph{semantically} reproducible:
running the same script on the same inputs with the same FontAnvil binary
should produce output with the same meaning.
\item It is not the policy of the FontAnvil project to make any attempt
at \emph{syntactic} reproducibility, it is not reasonable to expect
output to be byte for byte identical from one run to another, and in
general, the contrary should be expected.
\end{itemize}

\input{language.tex}
\chapter{Function and variable reference}

%%%%%%%%%%%%%%%%%%%%%%%%%%%%%%%%%%%%%%%%%%%%%%%%%%%%%%%%%%%%%%%%%%%%%%%%

\PEFuncRef{ATan2}

\texttt{ATan2(}\textit{y}, \textit{x}\texttt{)}

Returns the arctangent of $y/x$ in radians, using the signs of the arguments
to choose the quadrant.  \emph{Note the order of the arguments, with $y$
first.}  This function mimics the behaviour of the C math library
\texttt{atan2()} function.  This function may be used without a
loaded font.

%%%%%%%%%%%%%%%%%%%%%%%%%%%%%%%%%%%%

\PEFuncRef{AddAccent}

\texttt{AddAccent(}\textit{accent}, [\textit{pos-flags}]\texttt{)}

Perform one step in the process of building a composite glyph from a base
and an accent.  This involves three glyph slots:
\begin{itemize}
  \item The base glyph, which might be a letter, like e.  This must exist
    somewhere in the font.
  \item The glyph slot that will contain the composite, which might be an
    accented-letter glyph like \"{e}.  When \texttt{AddAccent()} is called,
    this slot must be selected and must be the only thing selected.  It must
    already contain a reference to the base glyph, and that must be the
    first reference.
  \item The glyph slot containing the accent itself.  This must be specified
    as the first argument to \texttt{AddAccent()}, either as a string which
    names the accent glyph slot or as a Unicode code point.
\end{itemize}

FontAnvil will attempt to place the accent in the composite glyph slot, in
an appropriate position relative to the base.  Without the second argument,
it will choose that position depending on the Unicode value of the accent. 
The second argument if specified may be one of the
following integer flags.  In principle, it could be a bitwise OR of more than
one of them, but that makes very little sense.

\begin{tabular}{r|l}
  flag & position \\ \hline
  0x100 & Above \\
  0x200 & Below \\
  0x400 & Overstrike \\
  0x800 & Left \\
  0x1000 & Right \\
  0x4000 & Center Left \\
  0x8000 & Center Right \\
  0x10000 & Centered Outside \\ 
  0x20000 & Outside \\
  0x40000 & Left Edge \\
  0x80000 & Right Edge \\
  0x100000  & Touching
\end{tabular}

%%%%%%%%%%%%%%%%%%%%%%%%%%%%%%%%%%%%

\PEFuncRef{AddAnchorClass}

\texttt{AddAnchorClass(}\textit{name}, \textit{type},
  \textit{subtable}\texttt{)}

Add an anchor class (for use with OpenType GPOS features) to the current
font.  All three arguments are required, and are strings.  The first is the
name of the class; the second is one of \texttt{"default"},
\texttt{"mk-mk"}, or \texttt{"cursive"}, indicating the general kind of
attachment (mark to base, mark to mark, or cursive, respectively); and the
third is the name of the lookup subtable in which the class will be used.

%%%%%%%%%%%%%%%%%%%%%%%%%%%%%%%%%%%%

\PEFuncRef{AddAnchorPoint}

\texttt{AddAnchorPoint(}\textit{name}, \textit{type}, \textit{x},
  \textit{y}, [\textit{lig-index}]\texttt{)}

Add an anchor point for mark attachment to the currently selected glyph. 
It is an error to attempt this if no, or more than one, glyph is selected. 
The \textit{name} argument specifies the class, such as was used in
\textit{AddAnchorClass()}.  The type should be one of \texttt{"mark"},
\texttt{"basechar"} (syn.\ \texttt{"base"}), \texttt{"baselig"} (syn.\
\texttt{"ligature"}), \texttt{"basemark"}, \texttt{"cursentry"} (syn.\
\texttt{"entry"}, \texttt{"cursexit"} (syn.\ \texttt{"exit"}), or
\texttt{"default"}.  The \texttt{"default"} choice attempts to guess an
appropriate type.  The \textit{x} and \textit{y} arguments specify the
coordinates of the anchor.  The final argument is required if and only if
the type is base ligature.

%%%%%%%%%%%%%%%%%%%%%%%%%%%%%%%%%%%%

\PEFuncRef{AddDHint}

\texttt{AddDHint(}$x_1$, $y_1$, $x_2$, $y_2$, $x_\mathrm{u}$,
  $y_\mathrm{u}$\texttt{)}

Add a diagonal hint to any selected glyphs.  A diagonal hint is specified by
three pairs of X-Y coordinates:  $(x_1,y_1)$ and $(x_2,y_2)$, which specify
two points on opposite sides of the diagonal stem, and
$(x_\mathrm{u},y_\mathrm{u})$, which
specifies a unit vector in the direction of the stem.

%%%%%%%%%%%%%%%%%%%%%%%%%%%%%%%%%%%%

\PEFuncRef{AddExtrema}

\texttt{AddExtrema(}[\textit{really}]\texttt{)}

Add extra points to spline segments (that is, break them into smaller
segments) in the selected glyph slots to ensure that segments achieve their
maximum and minimum X and Y values at endpoints and not in between. 
This is a technical requirement of some font formats.

If an extremum occurs very near but not at the endpoint of a segment, then
\texttt{AddExtrema()} will by default skip processing that extremum in order
to avoid creating a very short segment.  Specify a nonzero integer for the
optional argument \textit{really} to force adding extrema in such cases.

Rounding coordinates in a font to integer values will often cause the
splines to break the points-at-extrema rule, and adding points at extrema
will often break the integer-coordinates rule required by the same font
formats!  I do not know of a sequence of rounding, extrema-adding, and
simplification operations without manual intervention that will reliably
bring a font into compliance with both rules.

%%%%%%%%%%%%%%%%%%%%%%%%%%%%%%%%%%%%

\PEFuncRef{AddHHint}

\texttt{AddHHint(}\textit{start},\textit{width}\texttt{)}

Add a horizontal hint to any selected glyphs, starting at X coordinate
\textit{start} and extending for distance \textit{width}.

%%%%%%%%%%%%%%%%%%%%%%%%%%%%%%%%%%%%

\PEFuncRef{AddInstrs}

\texttt{AddInstrs(}\textit{dest}, \textit{replace}, \textit{instrs}\texttt{)}

Add Microsoft-style rasterization hints (``instructions'') to the font. 
These can be stored either in a TrueType table, or in an individual glyph,
as determined by the string argument \textit{dest}.  If \textit{dest} is
\texttt{"fpgm"} or \texttt{"prep"}, then the instructions will go in the
table of that name; if it is an empty string then the instructions will go
in any selected characters; and otherwise, the string value will be taken as
the name of the glyph where the instructions will go.  Glyphs actually named
``fpgm'' or ``prep'' apparently can only be specified by the selection
mechanism.

The \textit{replace} argument should be an integer; if it is nonzero then
the new instructions replace any currently in the specified destination, and
if it is zero then they are appended to any already present.  The
\textit{instrs} argument should be a string containing the instructions. 
This is code in a primitive assembly language for the TrueType instructions
bytecode interpreter.  The syntax defined by the TrueType standard appears
to be supported, with some undocumented but optional extensions proprietary
to FontForge (and inherited by FontAnvil).

%%%%%%%%%%%%%%%%%%%%%%%%%%%%%%%%%%%%

\PEFuncRef{AddLookup}

\texttt{AddLookup(}\textit{name}, \textit{type}, \textit{flags},
\textit{features}, [\textit{after}]\texttt{)}

Add a new lookup to the font.  The new lookup will be empty; other functions
can be used to put things in it.

The \textit{name} is a string.  The
\textit{type} is a string from the following list, identifying the algorithm
used for looking up glyphs or glyph sequences and the kind of substitution
or return value the lookup can produce: gpos\_context, gpos\_contextchain,
gpos\_cursive, gpos\_mark2base, gpos\_mark2ligature, gpos\_mark2mark,
gpos\_marktobase, gpos\_marktoligature, gpos\_marktomark, gpos\_pair,
gpos\_single, gsub\_alternate, gsub\_context, gsub\_contextchain,
gsub\_ligature, gsub\_multiple, gsub\_revesechain, gsub\_single,
kern\_statemachine, morx\_context, morx\_indic, morx\_insert.  The type
also implicitly specifies which table will contain the new lookup.

The \textit{flags} argument is an integer formed by adding these flag
values, which are equivalent to flag names in feature file syntax:
\begin{itemize}
  \item 1: RightToLeft;
  \item 2: IgnoreBaseGlyphs;
  \item 4: IgnoreLigatures; and
  \item 8: IgnoreMarks.
\end{itemize}

The \textit{features} argument expressed which OpenType features
will invoke this lookup.  It should be an array value, each of whose entries
is a two-element array, containing as its first element a four-letter string
identifying the feature, and as its second element another array of
four-letter strings listing the language systems in which that feature will
use this lookup.

The \textit{after} argument, if specified, causes the new lookup to be added
after the lookup with the specified name.  Otherwise, it will be added first
in its table.

%%%%%%%%%%%%%%%%%%%%%%%%%%%%%%%%%%%%

\PEFuncRef{AddLookupSubtable}

\texttt{AddLookupSubtable(}\textit{lookup}, \textit{name}, [\textit{after}]\texttt{)}

Add a new subtable to a lookup.  The lookup to use is specified as the first
argument, and the second is the name for the new subtable.  The optional
\textit{after} argument causes the new subtable to be added after the named
subtable; otherwise, it will be first in the lookup.

%%%%%%%%%%%%%%%%%%%%%%%%%%%%%%%%%%%%

\PEFuncRef{AddPosSub}

\texttt{AddPosSub(}\textit{arg}, \textit{arg}, \textit{arg}, [\textit{arg}], [\textit{arg}], [\textit{arg}], [\textit{arg}], [\textit{arg}], [\textit{arg}], [\textit{arg}]\texttt{)}

%%%%%%%%%%%%%%%%%%%%%%%%%%%%%%%%%%%%

\PEFuncRef{AddSizeFeature}

\texttt{AddSizeFeature(}\textit{arg}, \textit{arg}, [\textit{arg}], [\textit{arg}], [\textit{arg}]\texttt{)}

%%%%%%%%%%%%%%%%%%%%%%%%%%%%%%%%%%%%

\PEFuncRef{AddVHint}

\texttt{AddVHint(}\textit{start},\textit{width}\texttt{)}

Add a vertical hint to any selected glyphs, starting at Y coordinate
\textit{start} and extending for distance \textit{width}.

%%%%%%%%%%%%%%%%%%%%%%%%%%%%%%%%%%%%

\PEFuncRef{ApplySubstitution}

\texttt{ApplySubstitution(}\textit{script},\textit{lang},\textit{feature}\texttt{)}

Apply OpenType-style substitutions to selected glyphs.  All three arguments
should be strings of up to four characters, which will be blank-padded if
shorter; the \textit{script} and \textit{lang} tags (but not
\textit{feature}) may also have the wildcard value * (a single asterisk),
which matches everything.  FontAnvil will apply any substitution rules in
the current font for the selected glyphs that match the specified script,
language, and feature (or ignoring the checks for script or language if
applying the wildcard); glyph slots for which a substitution is found will
be replaced with the contents of the slots specified by the substitution
rules.

Context-dependent substitutions appear to be ignored, and it is not
specified what happens if the result of the subsitution rule is not exactly
one glyph.

%%%%%%%%%%%%%%%%%%%%%%%%%%%%%%%%%%%%

\PEFuncRef{Array}

\texttt{Array(}\textit{size}\texttt{)}

Creates and returns a value of array type, with the given number of
elements initialized to void.
This function may be used without a loaded font.

%%%%%%%%%%%%%%%%%%%%%%%%%%%%%%%%%%%%

\PEFuncRef{AskUser}

\texttt{AskUser(}\textit{prompt}, [\textit{default}]\texttt{)}

Asks the user a question.  The string \textit{prompt} is written out to
standard output, and the next line from standard input is captured to become
the return value.  A line is terminated by a newline character, which will
be included in the return value.  On error, or if the user enters a blank
line, the return value will be \textit{default} if specified, or an empty
string if not.
This function may be used without a loaded font.

If \FFdiff FontAnvil was compiled with readline support, then
\texttt{AskUser()} will allow command-line editing.  This is a
FontAnvil extension, not supported by FontForge.

%%%%%%%%%%%%%%%%%%%%%%%%%%%%%%%%%%%%

\PEFuncRef{AutoCounter}

\texttt{AutoCounter(}\texttt{)}

Automatically generate ``counter masks'' for selected glyphs.

%%%%%%%%%%%%%%%%%%%%%%%%%%%%%%%%%%%%

\PEFuncRef{AutoHint}

\texttt{AutoHint()}

Automatically generate Adobe-style rasterization hints for selected
glyphs.

%%%%%%%%%%%%%%%%%%%%%%%%%%%%%%%%%%%%

\PEFuncRef{AutoInstr}

\texttt{AutoInstr()}

Automatically generate Microsoft-style rasterization hints (that is,
``instructions'') for selected glyphs.

%%%%%%%%%%%%%%%%%%%%%%%%%%%%%%%%%%%%

\PEFuncRef{AutoKern}

\texttt{AutoKern(}\textit{arg}, \textit{arg}, \textit{arg}, \textit{arg}\texttt{)}

%%%%%%%%%%%%%%%%%%%%%%%%%%%%%%%%%%%%

\PEFuncRef{AutoTrace}

\texttt{AutoTrace()}

Automatically trace the background images of the selected glyphs into
outlines.  \FFdiff In FontForge, this function will only work if Autotrace
or Potrace is installed where the interpreter can find it.  FontAnvil uses
built-in code derived from Potrace instead of calling an external program.

%%%%%%%%%%%%%%%%%%%%%%%%%%%%%%%%%%%%

\PEFuncRef{AutoWidth}

\texttt{AutoWidth(}\textit{arg}, \textit{arg}, [\textit{arg}]\texttt{)}

%%%%%%%%%%%%%%%%%%%%%%%%%%%%%%%%%%%%

\PEFuncRef{BitmapsAvail}

\texttt{BitmapsAvail(} \textit{sizes}, [\textit{rasterized}]\ldots\texttt{)}

Choose the set of bitmap or greymap sizes (``strikes'') that will be stored
in the current font.  The \textit{sizes} argument should be a list of
integers; these are either plain numbers of pixels (implicitly specifying
one bit per pixel black and white bitmaps); or bit-field encoded numbers
specifying both a pixel size (in the least significant 16 bits) and a number
of bits per pixel (in the higher bits).  For instance, 0x8000C specifies a
12 pixel greymap font with eight bits per pixel.

If any sizes are specified that are not already in the font, they will be
added.  If any sizes are not specified but exist in the font, they will be
removed; so the final list of sizes will match what was specified.  Any
newly-created strikes will be filled in with rasterized versions of the
splines in the font, if \textit{rasterized} is not specified or is a nonzero
integer; the new strikes will be blank if \textit{rasterized} is zero.

%%%%%%%%%%%%%%%%%%%%%%%%%%%%%%%%%%%%

\PEFuncRef{BitmapsRegen}

\texttt{BitmapsRegen(} \textit{sizes}\texttt{)}

Render bitmaps from the splines in the current font.  The list of bitmap
sizes (and bit depths, in the case of greymap strikes) is specified as an
array in the same format as with \texttt{BitmapsAvail()}.  All the specified
sizes must already exist.  Their contents will be replaced with the results
of the rendering; any other existing sizes are not changed.

%%%%%%%%%%%%%%%%%%%%%%%%%%%%%%%%%%%%

\PEFuncRef{BuildAccented}

\texttt{BuildAccented()}

If any selected glyphs are accented letters (determined by Unicode code
points, according to George Williams's undocumented rules), then replace
them with composite glyphs formed by references to base and accent glyphs,
in an unspecified way.

%%%%%%%%%%%%%%%%%%%%%%%%%%%%%%%%%%%%

\PEFuncRef{BuildComposite}

\texttt{BuildComposite()}

If any selected glyphs fall into a category described as ``ligatures and
whatnot'' (determined by Unicode code points, according to George Williams's
undocumented rules), then replace them with composite glyphs formed by
references to other glyphs, in an unspecified way.

%%%%%%%%%%%%%%%%%%%%%%%%%%%%%%%%%%%%

\PEFuncRef{BuildDuplicate}

\texttt{BuildDuplicate()}

For all selected glyph slots, if the glyph slot is associated with a Unicode
code point that is an alternate of some other Unicode code point according
to undocumented rules, and there is a glyph in the other code point's slot,
then point the current glyph slot to that other glyph, thereby creating a
deprecated multi-slot glyph structure.  What happens to any former glyph in
the current slot is not clear; the code looks like it may leak the memory.

The intended use of this function seems to have been mainly to build
fonts with backward compatibility to certain Chinese encodings (perhaps
based on early versions of Unicode without full coverage) that placed glyphs
in the PUA for characters that are also (now) found at standard code points.

Do not use this function.  It is kept in FontAnvil (for the moment, quite
possibly to be removed some day) for backward compatibility, because some
scripts may depend on the built-in rules.  Because they are undocumented,
those are not easy to reproduce with other functions.  Creating multi-slot
glyph structures in general is deprecated in the relevant font standards and
so should rarely be attempted; and if you really need a multi-slot glyph
structure, the \texttt{SameGlyphAs()} function is a more controllable way to
build them.

%%%%%%%%%%%%%%%%%%%%%%%%%%%%%%%%%%%%

\PEFuncRef{CIDChangeSubFont}

\texttt{CIDChangeSubFont(}\textit{arg}\texttt{)}

%%%%%%%%%%%%%%%%%%%%%%%%%%%%%%%%%%%%

\PEFuncRef{CIDFlatten}

\texttt{CIDFlatten(}\texttt{)}

%%%%%%%%%%%%%%%%%%%%%%%%%%%%%%%%%%%%

\PEFuncRef{CIDFlattenByCMap}

\texttt{CIDFlattenByCMap(}\textit{arg}\texttt{)}

%%%%%%%%%%%%%%%%%%%%%%%%%%%%%%%%%%%%

\PEFuncRef{CIDSetFontNames}

\texttt{CIDSetFontNames(}\textit{arg}, \textit{arg}, [\textit{arg}], [\textit{arg}], [\textit{arg}], [\textit{arg}]\texttt{)}

%%%%%%%%%%%%%%%%%%%%%%%%%%%%%%%%%%%%

\PEFuncRef{CanonicalContours}

\texttt{CanonicalContours(}\texttt{)}

%%%%%%%%%%%%%%%%%%%%%%%%%%%%%%%%%%%%

\PEFuncRef{CanonicalStart}

\texttt{CanonicalStart(}\texttt{)}

%%%%%%%%%%%%%%%%%%%%%%%%%%%%%%%%%%%%

\PEFuncRef{Ceil}

\texttt{Ceil(}\textit{x}\texttt{)}

Returns $\lceil x \rceil$.  That is the ceiling of $x$, or the least
integer greater than or equal to $x$.  This function may be used without a
loaded font.

%%%%%%%%%%%%%%%%%%%%%%%%%%%%%%%%%%%%

\PEFuncRef{CenterInWidth}

\texttt{CenterInWidth(}\texttt{)}

%%%%%%%%%%%%%%%%%%%%%%%%%%%%%%%%%%%%

\PEFuncRef{ChangePrivateEntry}

\texttt{ChangePrivateEntry(}\textit{arg}, \textit{arg}\texttt{)}

%%%%%%%%%%%%%%%%%%%%%%%%%%%%%%%%%%%%

\PEFuncRef{ChangeWeight}

\texttt{ChangeWeight(}[\textit{arg}]\texttt{)}

%%%%%%%%%%%%%%%%%%%%%%%%%%%%%%%%%%%%

\PEFuncRef{CharCnt}

\texttt{CharCnt()}

Return the number of glyph slots in the current font, including both encoded
and unencoded slots.

%%%%%%%%%%%%%%%%%%%%%%%%%%%%%%%%%%%%

\PEFuncRef{CharInfo}

\texttt{CharInfo(}\textit{arg}, \textit{arg}\texttt{)}

%%%%%%%%%%%%%%%%%%%%%%%%%%%%%%%%%%%%

\PEFuncRef{CheckForAnchorClass}

\texttt{CheckForAnchorClass(}\textit{arg}\texttt{)}

%%%%%%%%%%%%%%%%%%%%%%%%%%%%%%%%%%%%

\PEFuncRef{Chr}

\texttt{Chr(}\textit{int}\texttt{)}

Takes a single integer in the range -128 to 255 and returns a string
containing that byte, folding negative values into two's complement
notation.

\noindent\texttt{Chr(}\textit{array}\texttt{)}

Takes an array of integers in the range -128 to 255 and returns a string
consisting of those bytes.  This is the inverse of the \texttt{Ord()}
function; see also \texttt{Utf8()} and \texttt{Ucs4()}, which operate on
Unicode strings and code points instead of byte values.

This function may be used without a loaded font.  Note that the integers are
\emph{byte} values.  Code points U+0080 and beyond may be constructed by
spelling them out in UTF-8.  It is also possible, but not recommended, to
construct strings that are invalid UTF-8.

Since strings are stored in null-terminated form internally, passing a zero,
although not reported as an error, will cause the string to terminate
immediately before the zero.  It is not possible to create a string value in
PE Script that meaningfully \emph{contains} a zero byte.

I \FFdiff have submitted a patch to FontForge, but as of the current writing
(June 2015), FontForge does not document its support of zero and
negative numbers,
and documents but does not correctly implement array-valued arguments. 
FontAnvil behaves as documented here.

%%%%%%%%%%%%%%%%%%%%%%%%%%%%%%%%%%%%

\PEFuncRef{Clear}

\texttt{Clear()}

%%%%%%%%%%%%%%%%%%%%%%%%%%%%%%%%%%%%

\PEFuncRef{ClearBackground}

\texttt{ClearBackground(}\texttt{)}

%%%%%%%%%%%%%%%%%%%%%%%%%%%%%%%%%%%%

\PEFuncRef{ClearCharCounterMasks}

\texttt{ClearCharCounterMasks(}\texttt{)}

%%%%%%%%%%%%%%%%%%%%%%%%%%%%%%%%%%%%

\PEFuncRef{ClearGlyphCounterMasks}

\texttt{ClearGlyphCounterMasks(}\texttt{)}

%%%%%%%%%%%%%%%%%%%%%%%%%%%%%%%%%%%%

\PEFuncRef{ClearHints}

\texttt{ClearHints(}[\textit{arg}]\texttt{)}

%%%%%%%%%%%%%%%%%%%%%%%%%%%%%%%%%%%%

\PEFuncRef{ClearInstrs}

\texttt{ClearInstrs(}\texttt{)}

%%%%%%%%%%%%%%%%%%%%%%%%%%%%%%%%%%%%

\PEFuncRef{ClearPrivateEntry}

\texttt{ClearPrivateEntry(}\textit{arg}\texttt{)}

%%%%%%%%%%%%%%%%%%%%%%%%%%%%%%%%%%%%

\PEFuncRef{ClearTable}

\texttt{ClearTable(}\textit{arg}\texttt{)}

%%%%%%%%%%%%%%%%%%%%%%%%%%%%%%%%%%%%

\PEFuncRef{Close}

\texttt{Close(}\texttt{)}

%%%%%%%%%%%%%%%%%%%%%%%%%%%%%%%%%%%%

\PEFuncRef{CompareFonts}

\texttt{CompareFonts(}\textit{arg}, \textit{arg}, \textit{arg}\texttt{)}

%%%%%%%%%%%%%%%%%%%%%%%%%%%%%%%%%%%%

\PEFuncRef{CompareGlyphs}

\texttt{CompareGlyphs(}[\textit{arg}], [\textit{arg}], [\textit{arg}], [\textit{arg}], [\textit{arg}], [\textit{arg}]\texttt{)}

%%%%%%%%%%%%%%%%%%%%%%%%%%%%%%%%%%%%

\PEFuncRef{ControlAfmLigatureOutput}

\texttt{ControlAfmLigatureOutput(}\textit{arg}, \textit{arg}\texttt{)}

%%%%%%%%%%%%%%%%%%%%%%%%%%%%%%%%%%%%

\PEFuncRef{ConvertByCMap}

\texttt{ConvertByCMap(}\textit{arg}\texttt{)}

%%%%%%%%%%%%%%%%%%%%%%%%%%%%%%%%%%%%

\PEFuncRef{ConvertToCID}

\texttt{ConvertToCID(}\textit{arg}, \textit{arg}, \textit{arg}\texttt{)}

%%%%%%%%%%%%%%%%%%%%%%%%%%%%%%%%%%%%

\PEFuncRef{Copy}

\texttt{Copy(}\texttt{)}

%%%%%%%%%%%%%%%%%%%%%%%%%%%%%%%%%%%%

\PEFuncRef{CopyAnchors}

\texttt{CopyAnchors(}\texttt{)}

%%%%%%%%%%%%%%%%%%%%%%%%%%%%%%%%%%%%

\PEFuncRef{CopyFgToBg}

\texttt{CopyFgToBg(}\texttt{)}

%%%%%%%%%%%%%%%%%%%%%%%%%%%%%%%%%%%%

\PEFuncRef{CopyGlyphFeatures}

\texttt{CopyGlyphFeatures(}, \ldots\texttt{)}

%%%%%%%%%%%%%%%%%%%%%%%%%%%%%%%%%%%%

\PEFuncRef{CopyLBearing}

\texttt{CopyLBearing(}\texttt{)}

%%%%%%%%%%%%%%%%%%%%%%%%%%%%%%%%%%%%

\PEFuncRef{CopyRBearing}

\texttt{CopyRBearing(}\texttt{)}

%%%%%%%%%%%%%%%%%%%%%%%%%%%%%%%%%%%%

\PEFuncRef{CopyReference}

\texttt{CopyReference(}\texttt{)}

%%%%%%%%%%%%%%%%%%%%%%%%%%%%%%%%%%%%

\PEFuncRef{CopyUnlinked}

\texttt{CopyUnlinked(}\texttt{)}

%%%%%%%%%%%%%%%%%%%%%%%%%%%%%%%%%%%%

\PEFuncRef{CopyVWidth}

\texttt{CopyVWidth(}\texttt{)}

%%%%%%%%%%%%%%%%%%%%%%%%%%%%%%%%%%%%

\PEFuncRef{CopyWidth}

\texttt{CopyWidth(}\texttt{)}

%%%%%%%%%%%%%%%%%%%%%%%%%%%%%%%%%%%%

\PEFuncRef{CorrectDirection}

\texttt{CorrectDirection(}\textit{arg}\texttt{)}

%%%%%%%%%%%%%%%%%%%%%%%%%%%%%%%%%%%%

\PEFuncRef{Cos}

\texttt{Cos(}\textit{theta}\texttt{)}

Returns the cosine of \textit{theta}, which is measured in radians.
This function may be used without a loaded font.

%%%%%%%%%%%%%%%%%%%%%%%%%%%%%%%%%%%%

\PEFuncRef{Cut}

\texttt{Cut(}\texttt{)}

%%%%%%%%%%%%%%%%%%%%%%%%%%%%%%%%%%%%

\PEFuncRef{DebugCrashFontForge}

\texttt{DebugCrashFontForge(}\ldots\texttt{)}

Causes FontAnvil [sic] to attempt to write to a null pointer, which should
crash the interpreter.  This may be of some use in debugging.  Arguments are
ignored.  Function name retained for compatibility.  In \FFdiff FontForge,
this function requires a loaded font, but that limitation is removed in
FontAnvil.

%%%%%%%%%%%%%%%%%%%%%%%%%%%%%%%%%%%%

\PEFuncRef{DefaultOtherSubrs}

\texttt{DefaultOtherSubrs(}\texttt{)}

Set the global store of ``other subroutines,'' which are fragments of
PostScript code used for special purposes within PostScript fonts, to its
default contents specified by Adobe.  This function may be used without a
loaded font.

%%%%%%%%%%%%%%%%%%%%%%%%%%%%%%%%%%%%

\PEFuncRef{DefaultRoundToGrid}

\texttt{DefaultRoundToGrid(}\texttt{)}

%%%%%%%%%%%%%%%%%%%%%%%%%%%%%%%%%%%%

\PEFuncRef{DefaultUseMyMetrics}

\texttt{DefaultUseMyMetrics(}\texttt{)}

%%%%%%%%%%%%%%%%%%%%%%%%%%%%%%%%%%%%

\PEFuncRef{DetachAndRemoveGlyphs}

\texttt{DetachAndRemoveGlyphs(}, \ldots\texttt{)}

%%%%%%%%%%%%%%%%%%%%%%%%%%%%%%%%%%%%

\PEFuncRef{DetachGlyphs}

\texttt{DetachGlyphs(}, \ldots\texttt{)}

%%%%%%%%%%%%%%%%%%%%%%%%%%%%%%%%%%%%

\PEFuncRef{DontAutoHint}

\texttt{DontAutoHint(}\texttt{)}

%%%%%%%%%%%%%%%%%%%%%%%%%%%%%%%%%%%%

\PEFuncRef{DrawsSomething}

\texttt{DrawsSomething(}\textit{arg}\texttt{)}

%%%%%%%%%%%%%%%%%%%%%%%%%%%%%%%%%%%%

\PEFuncRef{Error}

\texttt{Error(}\textit{msg}\texttt{)}

This function may be used without a loaded font.

%%%%%%%%%%%%%%%%%%%%%%%%%%%%%%%%%%%%

\PEFuncRef{Exp}

\texttt{Exp(}\textit{x}\texttt{)}

Compute the exponential function, $e^x$.
This function may be used without a loaded font.

%%%%%%%%%%%%%%%%%%%%%%%%%%%%%%%%%%%%

\PEFuncRef{ExpandStroke}

\texttt{ExpandStroke(}\textit{arg}, \textit{arg}, [\textit{arg}], [\textit{arg}], [\textit{arg}], [\textit{arg}]\texttt{)}

%%%%%%%%%%%%%%%%%%%%%%%%%%%%%%%%%%%%

\PEFuncRef{Export}

\texttt{Export(}\textit{arg}, \textit{arg}\texttt{)}

%%%%%%%%%%%%%%%%%%%%%%%%%%%%%%%%%%%%

\PEFuncRef{FileAccess}

\texttt{FileAccess(}\textit{arg}, \textit{arg}\texttt{)}

This function may be used without a loaded font.

%%%%%%%%%%%%%%%%%%%%%%%%%%%%%%%%%%%%

\PEFuncRef{FindIntersections}

\texttt{FindIntersections(}\texttt{)}

%%%%%%%%%%%%%%%%%%%%%%%%%%%%%%%%%%%%

\PEFuncRef{FindOrAddCvtIndex}

\texttt{FindOrAddCvtIndex(}\textit{arg}, \textit{arg}\texttt{)}

%%%%%%%%%%%%%%%%%%%%%%%%%%%%%%%%%%%%

\PEFuncRef{Floor}

\texttt{Floor(}\textit{arg}\texttt{)}

Returns $\lfloor x \rfloor$.  That is the floor of $x$, or the greatest
integer less than or equal to $x$.  This function may be used without a
loaded font.

%%%%%%%%%%%%%%%%%%%%%%%%%%%%%%%%%%%%

\PEFuncRef{FontImage}

\texttt{FontImage(}\textit{arg}, \textit{arg}, \textit{arg}, [\textit{arg}]\texttt{)}

%%%%%%%%%%%%%%%%%%%%%%%%%%%%%%%%%%%%

\PEFuncRef{FontsInFile}

\texttt{FontsInFile(}\textit{arg}\texttt{)}

This function may be used without a loaded font.

%%%%%%%%%%%%%%%%%%%%%%%%%%%%%%%%%%%%

\PEFuncRef{Generate}

\texttt{Generate(}\textit{arg}, \textit{arg}, [\textit{arg}], [\textit{arg}], [\textit{arg}], [\textit{arg}]\texttt{)}

%%%%%%%%%%%%%%%%%%%%%%%%%%%%%%%%%%%%

\PEFuncRef{GenerateFamily}

\texttt{GenerateFamily(}\textit{arg}, \textit{arg}, \textit{arg}, \textit{arg}\texttt{)}

%%%%%%%%%%%%%%%%%%%%%%%%%%%%%%%%%%%%

\PEFuncRef{GenerateFeatureFile}

\texttt{GenerateFeatureFile(}\textit{arg}, \textit{arg}\texttt{)}

%%%%%%%%%%%%%%%%%%%%%%%%%%%%%%%%%%%%

\PEFuncRef{GetAnchorPoints}

\texttt{GetAnchorPoints(}, \ldots\texttt{)}

%%%%%%%%%%%%%%%%%%%%%%%%%%%%%%%%%%%%

\PEFuncRef{GetCoverageCounts}

\texttt{GetCoverageCounts()}

Print (to standard output) a table listing all the built-in functions in the
interpreter and how many times each one has been called in the current run
of FontAnvil; this information may be useful in verifying the coverage of an
interpreter test suite.
This function may be used without a loaded font.  This \FFdiff function is a
FontAnvil extension and is not available in FontForge.  The details of its
operation may change in some future version.

%%%%%%%%%%%%%%%%%%%%%%%%%%%%%%%%%%%%

\PEFuncRef{GetCvtAt}

\texttt{GetCvtAt(}\textit{arg}\texttt{)}

%%%%%%%%%%%%%%%%%%%%%%%%%%%%%%%%%%%%

\PEFuncRef{GetEnv}

\texttt{GetEnv(}\textit{arg}\texttt{)}

Return the string value of an environment variable, or an empty string if
the variable requested does not exist.
This function may be used without a loaded font.
In \FFdiff FontForge, requesting a non-existent variable is an error.

%%%%%%%%%%%%%%%%%%%%%%%%%%%%%%%%%%%%

\PEFuncRef{GetFontBoundingBox}

\texttt{GetFontBoundingBox(}\texttt{)}

%%%%%%%%%%%%%%%%%%%%%%%%%%%%%%%%%%%%

\PEFuncRef{GetLookupInfo}

\texttt{GetLookupInfo(}\textit{arg}\texttt{)}

%%%%%%%%%%%%%%%%%%%%%%%%%%%%%%%%%%%%

\PEFuncRef{GetLookupOfSubtable}

\texttt{GetLookupOfSubtable(}\textit{arg}\texttt{)}

%%%%%%%%%%%%%%%%%%%%%%%%%%%%%%%%%%%%

\PEFuncRef{GetLookupSubtables}

\texttt{GetLookupSubtables(}\textit{arg}\texttt{)}

%%%%%%%%%%%%%%%%%%%%%%%%%%%%%%%%%%%%

\PEFuncRef{GetLookups}

\texttt{GetLookups(}\textit{arg}\texttt{)}

%%%%%%%%%%%%%%%%%%%%%%%%%%%%%%%%%%%%

\PEFuncRef{GetMaxpValue}

\texttt{GetMaxpValue(}\textit{arg}\texttt{)}

%%%%%%%%%%%%%%%%%%%%%%%%%%%%%%%%%%%%

\PEFuncRef{GetOS2Value}

\texttt{GetOS2Value(}, \ldots\texttt{)}

%%%%%%%%%%%%%%%%%%%%%%%%%%%%%%%%%%%%

\PEFuncRef{GetPosSub}

\texttt{GetPosSub(}\textit{arg}\texttt{)}

%%%%%%%%%%%%%%%%%%%%%%%%%%%%%%%%%%%%

\PEFuncRef{GetPref}

\texttt{GetPref(}\textit{arg}\texttt{)}

This function may be used without a loaded font.

%%%%%%%%%%%%%%%%%%%%%%%%%%%%%%%%%%%%

\PEFuncRef{GetPrivateEntry}

\texttt{GetPrivateEntry(}\textit{arg}\texttt{)}

%%%%%%%%%%%%%%%%%%%%%%%%%%%%%%%%%%%%

\PEFuncRef{GetSubtableOfAnchorClass}

\texttt{GetSubtableOfAnchorClass(}\textit{arg}\texttt{)}

%%%%%%%%%%%%%%%%%%%%%%%%%%%%%%%%%%%%

\PEFuncRef{GetTTFName}

\texttt{GetTTFName(}\textit{arg}, \textit{arg}\texttt{)}

%%%%%%%%%%%%%%%%%%%%%%%%%%%%%%%%%%%%

\PEFuncRef{GetTeXParam}

\texttt{GetTeXParam(}\textit{arg}\texttt{)}

%%%%%%%%%%%%%%%%%%%%%%%%%%%%%%%%%%%%

\PEFuncRef{GlyphInfo}

\texttt{GlyphInfo(}, \ldots\texttt{)}

%%%%%%%%%%%%%%%%%%%%%%%%%%%%%%%%%%%%

\PEFuncRef{HFlip}

\texttt{HFlip(}\textit{arg}\texttt{)}

%%%%%%%%%%%%%%%%%%%%%%%%%%%%%%%%%%%%

\PEFuncRef{HasPreservedTable}

\texttt{HasPreservedTable(}\textit{arg}\texttt{)}

%%%%%%%%%%%%%%%%%%%%%%%%%%%%%%%%%%%%

\PEFuncRef{HasPrivateEntry}

\texttt{HasPrivateEntry(}\textit{arg}\texttt{)}

%%%%%%%%%%%%%%%%%%%%%%%%%%%%%%%%%%%%

\PEFuncRef{Import}

\texttt{Import(}\textit{arg}, \textit{arg}, [\textit{arg}]\texttt{)}

%%%%%%%%%%%%%%%%%%%%%%%%%%%%%%%%%%%%

\PEFuncRef{InFont}

\texttt{InFont(}[\textit{arg}]\texttt{)}

%%%%%%%%%%%%%%%%%%%%%%%%%%%%%%%%%%%%

\PEFuncRef{Inline}

\texttt{Inline(}\textit{arg}, \textit{arg}\texttt{)}

%%%%%%%%%%%%%%%%%%%%%%%%%%%%%%%%%%%%

\PEFuncRef{Int}

\texttt{Int(}\textit{x}\texttt{)}

Converts a real number or Unicode code point value to an integer, by means
of the C compiler's type coercion.  In the case of real numbers, that means
$x$ will be rounded toward zero.  An integer argument will be returned
unchanged.
This function may be used without a loaded font.

%%%%%%%%%%%%%%%%%%%%%%%%%%%%%%%%%%%%

\PEFuncRef{InterpolateFonts}

\texttt{InterpolateFonts(}\textit{arg}, \textit{arg}, \textit{arg}\texttt{)}

%%%%%%%%%%%%%%%%%%%%%%%%%%%%%%%%%%%%

\PEFuncRef{IsAlNum}

\texttt{IsAlNum(}\textit{arg}\texttt{)}

This function may be used without a loaded font.

%%%%%%%%%%%%%%%%%%%%%%%%%%%%%%%%%%%%

\PEFuncRef{IsAlpha}

\texttt{IsAlpha(}\textit{arg}\texttt{)}

This function may be used without a loaded font.

%%%%%%%%%%%%%%%%%%%%%%%%%%%%%%%%%%%%

\PEFuncRef{IsDigit}

\texttt{IsDigit(}\textit{arg}\texttt{)}

This function may be used without a loaded font.

%%%%%%%%%%%%%%%%%%%%%%%%%%%%%%%%%%%%

\PEFuncRef{IsFinite}

\texttt{IsFinite(}\textit{arg}\texttt{)}

This function may be used without a loaded font.

%%%%%%%%%%%%%%%%%%%%%%%%%%%%%%%%%%%%

\PEFuncRef{IsHexDigit}

\texttt{IsHexDigit(}\textit{arg}\texttt{)}

This function may be used without a loaded font.

%%%%%%%%%%%%%%%%%%%%%%%%%%%%%%%%%%%%

\PEFuncRef{IsLower}

\texttt{IsLower(}\textit{arg}\texttt{)}

This function may be used without a loaded font.

%%%%%%%%%%%%%%%%%%%%%%%%%%%%%%%%%%%%

\PEFuncRef{IsNan}

\texttt{IsNan(}\textit{arg}\texttt{)}

This function may be used without a loaded font.

%%%%%%%%%%%%%%%%%%%%%%%%%%%%%%%%%%%%

\PEFuncRef{IsSpace}

\texttt{IsSpace(}\textit{arg}\texttt{)}

This function may be used without a loaded font.

%%%%%%%%%%%%%%%%%%%%%%%%%%%%%%%%%%%%

\PEFuncRef{IsUpper}

\texttt{IsUpper(}\textit{arg}\texttt{)}

This function may be used without a loaded font.

%%%%%%%%%%%%%%%%%%%%%%%%%%%%%%%%%%%%

\PEFuncRef{Italic}

\texttt{Italic(}[\textit{arg}], [\textit{arg}], [\textit{arg}], [\textit{arg}], [\textit{arg}], [\textit{arg}], [\textit{arg}], [\textit{arg}], [\textit{arg}]\texttt{)}

%%%%%%%%%%%%%%%%%%%%%%%%%%%%%%%%%%%%

\PEFuncRef{Join}

\texttt{Join(}\texttt{)}

%%%%%%%%%%%%%%%%%%%%%%%%%%%%%%%%%%%%

\PEFuncRef{LoadEncodingFile}

\texttt{LoadEncodingFile(}\textit{arg}, \textit{arg}\texttt{)}

This function may be used without a loaded font.

%%%%%%%%%%%%%%%%%%%%%%%%%%%%%%%%%%%%

\PEFuncRef{LoadNamelist}

\texttt{LoadNamelist(}\textit{arg}\texttt{)}

This function may be used without a loaded font.

%%%%%%%%%%%%%%%%%%%%%%%%%%%%%%%%%%%%

\PEFuncRef{LoadNamelistDir}

\texttt{LoadNamelistDir(}[\textit{arg}]\texttt{)}

This function may be used without a loaded font.

%%%%%%%%%%%%%%%%%%%%%%%%%%%%%%%%%%%%

\PEFuncRef{LoadStringFromFile}

\texttt{LoadStringFromFile(}\textit{arg}\texttt{)}

This function may be used without a loaded font.

%%%%%%%%%%%%%%%%%%%%%%%%%%%%%%%%%%%%

\PEFuncRef{LoadTableFromFile}

\texttt{LoadTableFromFile(}\textit{arg}, \textit{arg}\texttt{)}

%%%%%%%%%%%%%%%%%%%%%%%%%%%%%%%%%%%%

\PEFuncRef{Log}

\texttt{Log(}\textit{x}\texttt{)}

Returns $\ln x$, that is the natural (base $e$) logarithm.  This is an error
if $x$ is zero, negative, or not a number.
This function may be used without a loaded font.

%%%%%%%%%%%%%%%%%%%%%%%%%%%%%%%%%%%%

\PEFuncRef{LookupStoreLigatureInAfm}

\texttt{LookupStoreLigatureInAfm(}\textit{arg}, \textit{arg}\texttt{)}

%%%%%%%%%%%%%%%%%%%%%%%%%%%%%%%%%%%%

\PEFuncRef{MMAxisBounds}

\texttt{MMAxisBounds(}\textit{arg}\texttt{)}

%%%%%%%%%%%%%%%%%%%%%%%%%%%%%%%%%%%%

\PEFuncRef{MMAxisNames}

\texttt{MMAxisNames(}\texttt{)}

%%%%%%%%%%%%%%%%%%%%%%%%%%%%%%%%%%%%

\PEFuncRef{MMBlendToNewFont}

\texttt{MMBlendToNewFont(}, \ldots\texttt{)}

%%%%%%%%%%%%%%%%%%%%%%%%%%%%%%%%%%%%

\PEFuncRef{MMChangeInstance}

\texttt{MMChangeInstance(}\textit{arg}\texttt{)}

%%%%%%%%%%%%%%%%%%%%%%%%%%%%%%%%%%%%

\PEFuncRef{MMChangeWeight}

\texttt{MMChangeWeight(}, \ldots\texttt{)}

%%%%%%%%%%%%%%%%%%%%%%%%%%%%%%%%%%%%

\PEFuncRef{MMInstanceNames}

\texttt{MMInstanceNames(}\texttt{)}

%%%%%%%%%%%%%%%%%%%%%%%%%%%%%%%%%%%%

\PEFuncRef{MMWeightedName}

\texttt{MMWeightedName(}\texttt{)}

%%%%%%%%%%%%%%%%%%%%%%%%%%%%%%%%%%%%

\PEFuncRef{MakeLine}

\texttt{MakeLine(}, \ldots\texttt{)}

%%%%%%%%%%%%%%%%%%%%%%%%%%%%%%%%%%%%

\PEFuncRef{MergeFeature}

\texttt{MergeFeature(}\textit{arg}\texttt{)}

%%%%%%%%%%%%%%%%%%%%%%%%%%%%%%%%%%%%

\PEFuncRef{MergeFonts}

\texttt{MergeFonts(}\textit{arg}, \textit{arg}\texttt{)}

%%%%%%%%%%%%%%%%%%%%%%%%%%%%%%%%%%%%

\PEFuncRef{MergeKern}

\texttt{MergeKern(}\textit{arg}\texttt{)}

%%%%%%%%%%%%%%%%%%%%%%%%%%%%%%%%%%%%

\PEFuncRef{MergeLookupSubtables}

\texttt{MergeLookupSubtables(}\textit{arg}, \textit{arg}\texttt{)}

%%%%%%%%%%%%%%%%%%%%%%%%%%%%%%%%%%%%

\PEFuncRef{MergeLookups}

\texttt{MergeLookups(}\textit{arg}, \textit{arg}\texttt{)}

%%%%%%%%%%%%%%%%%%%%%%%%%%%%%%%%%%%%

\PEFuncRef{Move}

\texttt{Move(}\textit{arg}, \textit{arg}\texttt{)}

%%%%%%%%%%%%%%%%%%%%%%%%%%%%%%%%%%%%

\PEFuncRef{MoveReference}

\texttt{MoveReference(}\textit{arg}, \textit{arg}, \textit{arg}, \ldots\texttt{)}

%%%%%%%%%%%%%%%%%%%%%%%%%%%%%%%%%%%%

\PEFuncRef{MultipleEncodingsToReferences}

\texttt{MultipleEncodingsToReferences(}\texttt{)}

%%%%%%%%%%%%%%%%%%%%%%%%%%%%%%%%%%%%

\PEFuncRef{NameFromUnicode}

\texttt{NameFromUnicode(}\textit{arg}, \textit{arg}\texttt{)}

This function may be used without a loaded font.

%%%%%%%%%%%%%%%%%%%%%%%%%%%%%%%%%%%%

\PEFuncRef{NearlyHvCps}

\texttt{NearlyHvCps(}[\textit{arg}], [\textit{arg}]\texttt{)}

%%%%%%%%%%%%%%%%%%%%%%%%%%%%%%%%%%%%

\PEFuncRef{NearlyHvLines}

\texttt{NearlyHvLines(}[\textit{arg}]\texttt{)}

%%%%%%%%%%%%%%%%%%%%%%%%%%%%%%%%%%%%

\PEFuncRef{NearlyLines}

\texttt{NearlyLines(}[\textit{arg}]\texttt{)}

%%%%%%%%%%%%%%%%%%%%%%%%%%%%%%%%%%%%

\PEFuncRef{New}

\texttt{New(}\texttt{)}

This function may be used without a loaded font.

%%%%%%%%%%%%%%%%%%%%%%%%%%%%%%%%%%%%

\PEFuncRef{NonLinearTransform}

\texttt{NonLinearTransform(}\textit{xexp}, \textit{yexp}\texttt{)}

Transform all the X and Y coordinates in the selected glyphs according to
two expressions given as strings.  The expressions are defined in a language
unique to this function, which can be summarized as follows, from highest to
lowest precedence class, with evaluation from left to right within a class:
\begin{itemize}
  \item constant reals, constant integers, and the variables \texttt{x} and
    \texttt{y}
  \item \texttt{!}, \texttt{()}, \texttt{sin()}, \texttt{cos()},
    \texttt{tan()}, \texttt{log()}, \texttt{exp()}, \texttt{sqrt()},
    \texttt{abs()}, \texttt{rint()}, \texttt{floor()},
    \texttt{ceil()}
  \item \texttt{\textasciicircum}
  \item \texttt{*}, \texttt{/}, \texttt{\%}
  \item \texttt{+}, \texttt{-}
  \item \texttt{==}, \texttt{!=}, \texttt{>=}, \texttt{>}, \texttt{<=},
    \texttt{<=}, \texttt{<}
  \item \texttt{\&\&}, \texttt{||}
  \item \texttt{?~:} (C-style ternary question mark)
\end{itemize}

The FontForge \FFdiff code base supports this function, but only as a
compile-time option that is turned off by default.  Thus, most actual
installations of FontForge will not support it.  It is unconditionally
enabled in FontAnvil.  FontForge documents \texttt{x} and \texttt{y} as
being the only atomic values available in the \texttt{NonLinearTransform()}
expression language (no integers or reals), but that is obviously
incorrect.  FontForge also incorrectly documents \texttt{float()} as a
function available in the language, instead of \texttt{floor()}.

%%%%%%%%%%%%%%%%%%%%%%%%%%%%%%%%%%%%

\PEFuncRef{Open}

\texttt{Open(}\textit{arg}, \textit{arg}\texttt{)}

This function may be used without a loaded font.

%%%%%%%%%%%%%%%%%%%%%%%%%%%%%%%%%%%%

\PEFuncRef{Ord}

\texttt{Ord(}\textit{str}\texttt{)}

Converts a string to an array of integers in the range 1 to 255,
representing the byte values in the string.  Note 0 is not included
because PE Script strings cannot contain zero bytes.
This is the inverse of the \texttt{Chr()}
function; see also \texttt{Utf8()} and \texttt{Ucs4()}, which operate on
Unicode strings and code points instead of byte values.

\texttt{Ord(}\textit{str}, \textit{pos}\texttt{)}

With the optional \textit{pos} argument, \texttt{Ord()} returns a single
integer instead of an array, representing the byte value at the given
zero-based index into the string.

This function may be used without a loaded font.

%%%%%%%%%%%%%%%%%%%%%%%%%%%%%%%%%%%%

\PEFuncRef{Outline}

\texttt{Outline(}\textit{arg}\texttt{)}

%%%%%%%%%%%%%%%%%%%%%%%%%%%%%%%%%%%%

\PEFuncRef{OverlapIntersect}

\texttt{OverlapIntersect(}\texttt{)}

%%%%%%%%%%%%%%%%%%%%%%%%%%%%%%%%%%%%

\PEFuncRef{Paste}

\texttt{Paste(}\texttt{)}

%%%%%%%%%%%%%%%%%%%%%%%%%%%%%%%%%%%%

\PEFuncRef{PasteInto}

\texttt{PasteInto(}\texttt{)}

%%%%%%%%%%%%%%%%%%%%%%%%%%%%%%%%%%%%

\PEFuncRef{PasteWithOffset}

\texttt{PasteWithOffset(}\textit{arg}, \textit{arg}\texttt{)}

%%%%%%%%%%%%%%%%%%%%%%%%%%%%%%%%%%%%

\PEFuncRef{PositionReference}

\texttt{PositionReference(}\textit{arg}, \textit{arg}, \textit{arg}, \ldots\texttt{)}

%%%%%%%%%%%%%%%%%%%%%%%%%%%%%%%%%%%%

\PEFuncRef{PostNotice}

\texttt{PostNotice(}\textit{arg}\texttt{)}

This function may be used without a loaded font.

%%%%%%%%%%%%%%%%%%%%%%%%%%%%%%%%%%%%

\PEFuncRef{Pow}

\texttt{Pow(}\textit{arg}, \textit{arg}\texttt{)}

This function may be used without a loaded font.

%%%%%%%%%%%%%%%%%%%%%%%%%%%%%%%%%%%%

\PEFuncRef{PreloadCidmap}

\texttt{PreloadCidmap(}\textit{arg}, \textit{arg}, \textit{arg}, \textit{arg}\texttt{)}

This function may be used without a loaded font.

%%%%%%%%%%%%%%%%%%%%%%%%%%%%%%%%%%%%

\PEFuncRef{Print}

\texttt{Print(}, \ldots\texttt{)}

This function may be used without a loaded font.

%%%%%%%%%%%%%%%%%%%%%%%%%%%%%%%%%%%%

\PEFuncRef{PrintFont}

\texttt{PrintFont(}\textit{arg}, \textit{arg}, [\textit{arg}], [\textit{arg}]\texttt{)}

%%%%%%%%%%%%%%%%%%%%%%%%%%%%%%%%%%%%

\PEFuncRef{PrintSetup}

\texttt{PrintSetup(}\textit{arg}, \textit{arg}, [\textit{arg}], [\textit{arg}]\texttt{)}

This function may be used without a loaded font.

%%%%%%%%%%%%%%%%%%%%%%%%%%%%%%%%%%%%

\PEFuncRef{PrivateGuess}

\texttt{PrivateGuess(}\textit{arg}\texttt{)}

%%%%%%%%%%%%%%%%%%%%%%%%%%%%%%%%%%%%

\PEFuncRef{Quit}

\texttt{Quit(}[\textit{arg}]\texttt{)}

This function may be used without a loaded font.

%%%%%%%%%%%%%%%%%%%%%%%%%%%%%%%%%%%%

\PEFuncRef{Rand}

\texttt{Rand()}

Return something called ``a random integer.''  FontForge \FFdiff does not
document what that means.  In fact, in FontForge it means whatever comes out
of a call to the C library \texttt{rand()} function, and what that actually
is will vary depending on the host platform.  FontAnvil uses a bundled copy
of the SIMD-oriented Fast Mersenne Twister by Saito and Matsumoto.  The
return values are uniformly distributed over the integers 0 to $2^{31}-1$,
that is 0 to 2147483647 (32-bit integers from the generator with the high
bit forced to zero to prevent signedness problems).  The generator is seeded
from the system clock and process ID when the interpreter starts.  The same
generator instance is used throughout FontAnvil wherever random numbers are
called for, including in operations that do not obviously involve
randomization (as a result of UUID generation and such).  Thus, scripts
should not depend on its producing a consistent sequence of numbers.

This function may be used without a loaded font.  See \texttt{RandReal()}
for a related function that may be of use.

%%%%%%%%%%%%%%%%%%%%%%%%%%%%%%%%%%%%

\PEFuncRef{RandReal}

\texttt{RandReal()}

Return a pseudorandom real number in the interval $[0,1)$.  This uses the
same underlying generator as \texttt{Rand()}, which see; but its output
range may be more convenient.  In principle, \texttt{RandReal()} may also
have slightly better precision, because it uses all 32 bits of the PRNG
output.  This \FFdiff function is a FontAnvil extension and is not available
in FontForge.  This function may be used without a loaded font.

%%%%%%%%%%%%%%%%%%%%%%%%%%%%%%%%%%%%

\PEFuncRef{ReadOtherSubrsFile}

\texttt{ReadOtherSubrsFile(}\textit{arg}\texttt{)}

This function may be used without a loaded font.

%%%%%%%%%%%%%%%%%%%%%%%%%%%%%%%%%%%%

\PEFuncRef{Real}

\texttt{Real(}\textit{arg}\texttt{)}

This function may be used without a loaded font.

%%%%%%%%%%%%%%%%%%%%%%%%%%%%%%%%%%%%

\PEFuncRef{Reencode}

\texttt{Reencode(}\textit{arg}, \textit{arg}\texttt{)}

%%%%%%%%%%%%%%%%%%%%%%%%%%%%%%%%%%%%

\PEFuncRef{RemoveAllKerns}

\texttt{RemoveAllKerns(}\texttt{)}

%%%%%%%%%%%%%%%%%%%%%%%%%%%%%%%%%%%%

\PEFuncRef{RemoveAllVKerns}

\texttt{RemoveAllVKerns(}\texttt{)}

%%%%%%%%%%%%%%%%%%%%%%%%%%%%%%%%%%%%

\PEFuncRef{RemoveAnchorClass}

\texttt{RemoveAnchorClass(}\textit{arg}\texttt{)}

%%%%%%%%%%%%%%%%%%%%%%%%%%%%%%%%%%%%

\PEFuncRef{RemoveDetachedGlyphs}

\texttt{RemoveDetachedGlyphs(}, \ldots\texttt{)}

%%%%%%%%%%%%%%%%%%%%%%%%%%%%%%%%%%%%

\PEFuncRef{RemoveLookup}

\texttt{RemoveLookup(}\textit{arg}, \textit{arg}\texttt{)}

%%%%%%%%%%%%%%%%%%%%%%%%%%%%%%%%%%%%

\PEFuncRef{RemoveLookupSubtable}

\texttt{RemoveLookupSubtable(}\textit{arg}, \textit{arg}\texttt{)}

%%%%%%%%%%%%%%%%%%%%%%%%%%%%%%%%%%%%

\PEFuncRef{RemoveOverlap}

\texttt{RemoveOverlap(}\texttt{)}

%%%%%%%%%%%%%%%%%%%%%%%%%%%%%%%%%%%%

\PEFuncRef{RemovePosSub}

\texttt{RemovePosSub(}\textit{arg}\texttt{)}

%%%%%%%%%%%%%%%%%%%%%%%%%%%%%%%%%%%%

\PEFuncRef{RemovePreservedTable}

\texttt{RemovePreservedTable(}\textit{arg}\texttt{)}

%%%%%%%%%%%%%%%%%%%%%%%%%%%%%%%%%%%%

\PEFuncRef{RenameGlyphs}

\texttt{RenameGlyphs(}\textit{arg}\texttt{)}

%%%%%%%%%%%%%%%%%%%%%%%%%%%%%%%%%%%%

\PEFuncRef{ReplaceCharCounterMasks}

\texttt{ReplaceCharCounterMasks(}\textit{arg}\texttt{)}

%%%%%%%%%%%%%%%%%%%%%%%%%%%%%%%%%%%%

\PEFuncRef{ReplaceCvtAt}

\texttt{ReplaceCvtAt(}\textit{arg}, \textit{arg}\texttt{)}

%%%%%%%%%%%%%%%%%%%%%%%%%%%%%%%%%%%%

\PEFuncRef{ReplaceGlyphCounterMasks}

\texttt{ReplaceGlyphCounterMasks(}\textit{arg}\texttt{)}

%%%%%%%%%%%%%%%%%%%%%%%%%%%%%%%%%%%%

\PEFuncRef{ReplaceWithReference}

\texttt{ReplaceWithReference(}[\textit{arg}], [\textit{arg}]\texttt{)}

%%%%%%%%%%%%%%%%%%%%%%%%%%%%%%%%%%%%

\PEFuncRef{Revert}

\texttt{Revert()}

Reload the current font from its file on disk (which must exist).

%%%%%%%%%%%%%%%%%%%%%%%%%%%%%%%%%%%%

\PEFuncRef{RevertToBackup}

\texttt{RevertToBackup()}

Reload the current font from its backup file (the one ending in
\textasciitilde, not a numbered revision), if one exists.  This is only
meaningful for fonts that are SFD files for which a backup was created by a
previous \texttt{Save()}.

%%%%%%%%%%%%%%%%%%%%%%%%%%%%%%%%%%%%

\PEFuncRef{Rotate}

\texttt{Rotate(}\textit{arg}, [\textit{arg}], [\textit{arg}]\texttt{)}

%%%%%%%%%%%%%%%%%%%%%%%%%%%%%%%%%%%%

\PEFuncRef{Round}

\texttt{Round(}\textit{arg}\texttt{)}

Returns the nearest integer to $x$, with ties broken by returning the
nearest \emph{even} integer.  This behaviour may differ on nonstandard
systems that lack the \texttt{fesetround()} library function.
This function may be used without a loaded font.

%%%%%%%%%%%%%%%%%%%%%%%%%%%%%%%%%%%%

\PEFuncRef{RoundToCluster}

\texttt{RoundToCluster(}[\textit{arg}], [\textit{arg}]\texttt{)}

%%%%%%%%%%%%%%%%%%%%%%%%%%%%%%%%%%%%

\PEFuncRef{RoundToInt}

\texttt{RoundToInt(}\textit{arg}\texttt{)}

Round coordinates in selected glyphs.  See
\hyperref[func:Round]{\texttt{Round()}} for rounding a
real number to an integer.

%%%%%%%%%%%%%%%%%%%%%%%%%%%%%%%%%%%%

\PEFuncRef{SameGlyphAs}

\texttt{SameGlyphAs(}\texttt{)}

%%%%%%%%%%%%%%%%%%%%%%%%%%%%%%%%%%%%

\PEFuncRef{Save}

\texttt{Save(}[\textit{arg}], [\textit{arg}]\texttt{)}

%%%%%%%%%%%%%%%%%%%%%%%%%%%%%%%%%%%%

\PEFuncRef{SaveTableToFile}

\texttt{SaveTableToFile(}\textit{arg}, \textit{arg}\texttt{)}

%%%%%%%%%%%%%%%%%%%%%%%%%%%%%%%%%%%%

\PEFuncRef{Scale}

\texttt{Scale(}\textit{arg}, \textit{arg}, [\textit{arg}], [\textit{arg}]\texttt{)}

%%%%%%%%%%%%%%%%%%%%%%%%%%%%%%%%%%%%

\PEFuncRef{ScaleToEm}

\texttt{ScaleToEm(}\textit{arg}, \textit{arg}\texttt{)}

%%%%%%%%%%%%%%%%%%%%%%%%%%%%%%%%%%%%

\PEFuncRef{Select}

\texttt{Select(}, \ldots\texttt{)}

%%%%%%%%%%%%%%%%%%%%%%%%%%%%%%%%%%%%

\PEFuncRef{SelectAll}

\texttt{SelectAll(}\texttt{)}

%%%%%%%%%%%%%%%%%%%%%%%%%%%%%%%%%%%%

\PEFuncRef{SelectAllInstancesOf}

\texttt{SelectAllInstancesOf(}, \ldots\texttt{)}

%%%%%%%%%%%%%%%%%%%%%%%%%%%%%%%%%%%%

\PEFuncRef{SelectBitmap}

\texttt{SelectBitmap(}\textit{arg}\texttt{)}

%%%%%%%%%%%%%%%%%%%%%%%%%%%%%%%%%%%%

\PEFuncRef{SelectByColor}

\texttt{SelectByColor(}\textit{arg}\texttt{)}

%%%%%%%%%%%%%%%%%%%%%%%%%%%%%%%%%%%%

\PEFuncRef{SelectByColour}

\texttt{SelectByColour(}\textit{arg}\texttt{)}

%%%%%%%%%%%%%%%%%%%%%%%%%%%%%%%%%%%%

\PEFuncRef{SelectByPosSub}

\texttt{SelectByPosSub(}\textit{arg}, \textit{arg}\texttt{)}

%%%%%%%%%%%%%%%%%%%%%%%%%%%%%%%%%%%%

\PEFuncRef{SelectChanged}

\texttt{SelectChanged(}\textit{arg}\texttt{)}

%%%%%%%%%%%%%%%%%%%%%%%%%%%%%%%%%%%%

\PEFuncRef{SelectFewer}

\texttt{SelectFewer(}\textit{arg}, \ldots\texttt{)}

%%%%%%%%%%%%%%%%%%%%%%%%%%%%%%%%%%%%

\PEFuncRef{SelectFewerSingletons}

\texttt{SelectFewerSingletons(}\textit{arg}, \ldots\texttt{)}

%%%%%%%%%%%%%%%%%%%%%%%%%%%%%%%%%%%%

\PEFuncRef{SelectGlyphsBoth}

\texttt{SelectGlyphsBoth(}\textit{arg}\texttt{)}

%%%%%%%%%%%%%%%%%%%%%%%%%%%%%%%%%%%%

\PEFuncRef{SelectGlyphsReferences}

\texttt{SelectGlyphsReferences(}\textit{arg}\texttt{)}

%%%%%%%%%%%%%%%%%%%%%%%%%%%%%%%%%%%%

\PEFuncRef{SelectGlyphsSplines}

\texttt{SelectGlyphsSplines(}\textit{arg}\texttt{)}

%%%%%%%%%%%%%%%%%%%%%%%%%%%%%%%%%%%%

\PEFuncRef{SelectHintingNeeded}

\texttt{SelectHintingNeeded(}\textit{arg}\texttt{)}

%%%%%%%%%%%%%%%%%%%%%%%%%%%%%%%%%%%%

\PEFuncRef{SelectIf}

\texttt{SelectIf(}, \ldots\texttt{)}

%%%%%%%%%%%%%%%%%%%%%%%%%%%%%%%%%%%%

\PEFuncRef{SelectInvert}

\texttt{SelectInvert(}\texttt{)}

%%%%%%%%%%%%%%%%%%%%%%%%%%%%%%%%%%%%

\PEFuncRef{SelectMore}

\texttt{SelectMore(}\textit{arg}, \ldots\texttt{)}

%%%%%%%%%%%%%%%%%%%%%%%%%%%%%%%%%%%%

\PEFuncRef{SelectMoreIf}

\texttt{SelectMoreIf(}\textit{arg}, \ldots\texttt{)}

%%%%%%%%%%%%%%%%%%%%%%%%%%%%%%%%%%%%

\PEFuncRef{SelectMoreSingletons}

\texttt{SelectMoreSingletons(}\textit{arg}, \ldots\texttt{)}

%%%%%%%%%%%%%%%%%%%%%%%%%%%%%%%%%%%%

\PEFuncRef{SelectMoreSingletonsIf}

\texttt{SelectMoreSingletonsIf(}\textit{arg}, \ldots\texttt{)}

%%%%%%%%%%%%%%%%%%%%%%%%%%%%%%%%%%%%

\PEFuncRef{SelectNone}

\texttt{SelectNone(}\texttt{)}

%%%%%%%%%%%%%%%%%%%%%%%%%%%%%%%%%%%%

\PEFuncRef{SelectSingletons}

\texttt{SelectSingletons(}, \ldots\texttt{)}

%%%%%%%%%%%%%%%%%%%%%%%%%%%%%%%%%%%%

\PEFuncRef{SelectSingletonsIf}

\texttt{SelectSingletonsIf(}, \ldots\texttt{)}

%%%%%%%%%%%%%%%%%%%%%%%%%%%%%%%%%%%%

\PEFuncRef{SelectWorthOutputting}

\texttt{SelectWorthOutputting(}\textit{arg}\texttt{)}

%%%%%%%%%%%%%%%%%%%%%%%%%%%%%%%%%%%%

\PEFuncRef{SetCharCnt}

\texttt{SetCharCnt(}\textit{arg}\texttt{)}

%%%%%%%%%%%%%%%%%%%%%%%%%%%%%%%%%%%%

\PEFuncRef{SetCharColor}

\texttt{SetCharColor(}\textit{arg}\texttt{)}

%%%%%%%%%%%%%%%%%%%%%%%%%%%%%%%%%%%%

\PEFuncRef{SetCharComment}

\texttt{SetCharComment(}\textit{arg}\texttt{)}

%%%%%%%%%%%%%%%%%%%%%%%%%%%%%%%%%%%%

\PEFuncRef{SetCharCounterMask}

\texttt{SetCharCounterMask(}\textit{arg}, \textit{arg}, \ldots\texttt{)}

%%%%%%%%%%%%%%%%%%%%%%%%%%%%%%%%%%%%

\PEFuncRef{SetCharName}

\texttt{SetCharName(}\textit{arg}, \textit{arg}\texttt{)}

%%%%%%%%%%%%%%%%%%%%%%%%%%%%%%%%%%%%

\PEFuncRef{SetFeatureList}

\texttt{SetFeatureList(}\textit{arg}, \textit{arg}\texttt{)}

%%%%%%%%%%%%%%%%%%%%%%%%%%%%%%%%%%%%

\PEFuncRef{SetFondName}

\texttt{SetFondName(}\textit{arg}\texttt{)}

%%%%%%%%%%%%%%%%%%%%%%%%%%%%%%%%%%%%

\PEFuncRef{SetFontHasVerticalMetrics}

\texttt{SetFontHasVerticalMetrics(}\textit{arg}\texttt{)}

%%%%%%%%%%%%%%%%%%%%%%%%%%%%%%%%%%%%

\PEFuncRef{SetFontNames}

\texttt{SetFontNames(}\textit{arg}, \textit{arg}, [\textit{arg}], [\textit{arg}], [\textit{arg}], [\textit{arg}]\texttt{)}

%%%%%%%%%%%%%%%%%%%%%%%%%%%%%%%%%%%%

\PEFuncRef{SetFontOrder}

\texttt{SetFontOrder(}\textit{arg}\texttt{)}

%%%%%%%%%%%%%%%%%%%%%%%%%%%%%%%%%%%%

\PEFuncRef{SetGasp}

\texttt{SetGasp(}, \ldots\texttt{)}

%%%%%%%%%%%%%%%%%%%%%%%%%%%%%%%%%%%%

\PEFuncRef{SetGlyphChanged}

\texttt{SetGlyphChanged(}\textit{arg}\texttt{)}

%%%%%%%%%%%%%%%%%%%%%%%%%%%%%%%%%%%%

\PEFuncRef{SetGlyphClass}

\texttt{SetGlyphClass(}\textit{arg}\texttt{)}

%%%%%%%%%%%%%%%%%%%%%%%%%%%%%%%%%%%%

\PEFuncRef{SetGlyphColor}

\texttt{SetGlyphColor(}\textit{arg}\texttt{)}

%%%%%%%%%%%%%%%%%%%%%%%%%%%%%%%%%%%%

\PEFuncRef{SetGlyphComment}

\texttt{SetGlyphComment(}\textit{arg}\texttt{)}

%%%%%%%%%%%%%%%%%%%%%%%%%%%%%%%%%%%%

\PEFuncRef{SetGlyphCounterMask}

\texttt{SetGlyphCounterMask(}, \ldots\texttt{)}

%%%%%%%%%%%%%%%%%%%%%%%%%%%%%%%%%%%%

\PEFuncRef{SetGlyphName}

\texttt{SetGlyphName(}, \ldots\texttt{)}

%%%%%%%%%%%%%%%%%%%%%%%%%%%%%%%%%%%%

\PEFuncRef{SetGlyphTeX}

\texttt{SetGlyphTeX(}\textit{arg}, \textit{arg}, \textit{arg}, [\textit{arg}]\texttt{)}

%%%%%%%%%%%%%%%%%%%%%%%%%%%%%%%%%%%%

\PEFuncRef{SetItalicAngle}

\texttt{SetItalicAngle(}\textit{arg}, \textit{arg}\texttt{)}

%%%%%%%%%%%%%%%%%%%%%%%%%%%%%%%%%%%%

\PEFuncRef{SetKern}

\texttt{SetKern(}\textit{arg}, \textit{arg}, \textit{arg}\texttt{)}

%%%%%%%%%%%%%%%%%%%%%%%%%%%%%%%%%%%%

\PEFuncRef{SetLBearing}

\texttt{SetLBearing(}\textit{arg}, \textit{arg}\texttt{)}

%%%%%%%%%%%%%%%%%%%%%%%%%%%%%%%%%%%%

\PEFuncRef{SetMacStyle}

\texttt{SetMacStyle(}\textit{arg}\texttt{)}

%%%%%%%%%%%%%%%%%%%%%%%%%%%%%%%%%%%%

\PEFuncRef{SetMaxpValue}

\texttt{SetMaxpValue(}\textit{arg}, \textit{arg}\texttt{)}

%%%%%%%%%%%%%%%%%%%%%%%%%%%%%%%%%%%%

\PEFuncRef{SetOS2Value}

\texttt{SetOS2Value(}, \ldots\texttt{)}

%%%%%%%%%%%%%%%%%%%%%%%%%%%%%%%%%%%%

\PEFuncRef{SetPanose}

\texttt{SetPanose(}\textit{arg}, \textit{arg}\texttt{)}

%%%%%%%%%%%%%%%%%%%%%%%%%%%%%%%%%%%%

\PEFuncRef{SetPref}

\texttt{SetPref(}\textit{arg}, \textit{arg}, \textit{arg}\texttt{)}

This function may be used without a loaded font.

%%%%%%%%%%%%%%%%%%%%%%%%%%%%%%%%%%%%

\PEFuncRef{SetRBearing}

\texttt{SetRBearing(}\textit{arg}, \textit{arg}\texttt{)}

%%%%%%%%%%%%%%%%%%%%%%%%%%%%%%%%%%%%

\PEFuncRef{SetTTFName}

\texttt{SetTTFName(}\textit{arg}, \textit{arg}, \textit{arg}\texttt{)}

%%%%%%%%%%%%%%%%%%%%%%%%%%%%%%%%%%%%

\PEFuncRef{SetTeXParams}

\texttt{SetTeXParams(}, \ldots\texttt{)}

%%%%%%%%%%%%%%%%%%%%%%%%%%%%%%%%%%%%

\PEFuncRef{SetUnicodeValue}

\texttt{SetUnicodeValue(}\textit{arg}, \textit{arg}\texttt{)}

%%%%%%%%%%%%%%%%%%%%%%%%%%%%%%%%%%%%

\PEFuncRef{SetUniqueID}

\texttt{SetUniqueID(}\textit{arg}\texttt{)}

%%%%%%%%%%%%%%%%%%%%%%%%%%%%%%%%%%%%

\PEFuncRef{SetVKern}

\texttt{SetVKern(}\textit{arg}, \textit{arg}, \textit{arg}\texttt{)}

%%%%%%%%%%%%%%%%%%%%%%%%%%%%%%%%%%%%

\PEFuncRef{SetVWidth}

\texttt{SetVWidth(}\textit{arg}, \textit{arg}\texttt{)}

%%%%%%%%%%%%%%%%%%%%%%%%%%%%%%%%%%%%

\PEFuncRef{SetWidth}

\texttt{SetWidth(}\textit{arg}, \textit{arg}\texttt{)}

%%%%%%%%%%%%%%%%%%%%%%%%%%%%%%%%%%%%

\PEFuncRef{Shadow}

\texttt{Shadow(}\textit{arg}, \textit{arg}, \textit{arg}\texttt{)}

%%%%%%%%%%%%%%%%%%%%%%%%%%%%%%%%%%%%

\PEFuncRef{Shell}

\texttt{Shell(}\textit{command}\texttt{)}

Execute a command in the system shell, via the C library \texttt{system(3)}
call.  Returns the integer return value from doing so.  All the usual
security considerations associated with shell interfaces apply.
This function may be used without a loaded font.  This \FFdiff function is a
FontAnvil extension and is not available in FontForge.

%%%%%%%%%%%%%%%%%%%%%%%%%%%%%%%%%%%%

\PEFuncRef{Simplify}

\texttt{Simplify(}, \ldots\texttt{)}

%%%%%%%%%%%%%%%%%%%%%%%%%%%%%%%%%%%%

\PEFuncRef{Sin}

\texttt{Sin(}\textit{theta}\texttt{)}

Returns the sine of \textit{theta}, which is measured in radians.
This function may be used without a loaded font.

%%%%%%%%%%%%%%%%%%%%%%%%%%%%%%%%%%%%

\PEFuncRef{SizeOf}

\texttt{SizeOf(}\textit{arg}\texttt{)}

This function may be used without a loaded font.

%%%%%%%%%%%%%%%%%%%%%%%%%%%%%%%%%%%%

\PEFuncRef{Skew}

\texttt{Skew(}\textit{arg}, \textit{arg}, [\textit{arg}], [\textit{arg}]\texttt{)}

%%%%%%%%%%%%%%%%%%%%%%%%%%%%%%%%%%%%

\PEFuncRef{SmallCaps}

\texttt{SmallCaps(}[\textit{arg}], [\textit{arg}], [\textit{arg}], [\textit{arg}]\texttt{)}

%%%%%%%%%%%%%%%%%%%%%%%%%%%%%%%%%%%%

\PEFuncRef{Sqrt}

\texttt{Sqrt(}\textit{arg}\texttt{)}

This function may be used without a loaded font.

%%%%%%%%%%%%%%%%%%%%%%%%%%%%%%%%%%%%

\PEFuncRef{StrJoin}

\texttt{StrJoin(}\textit{arg}, \textit{arg}\texttt{)}

This function may be used without a loaded font.

%%%%%%%%%%%%%%%%%%%%%%%%%%%%%%%%%%%%

\PEFuncRef{StrSplit}

\texttt{StrSplit(}\textit{arg}, \textit{arg}, \textit{arg}\texttt{)}

This function may be used without a loaded font.

%%%%%%%%%%%%%%%%%%%%%%%%%%%%%%%%%%%%

\PEFuncRef{Strcasecmp}

\texttt{Strcasecmp(}\textit{arg}, \textit{arg}\texttt{)}

This function may be used without a loaded font.

%%%%%%%%%%%%%%%%%%%%%%%%%%%%%%%%%%%%

\PEFuncRef{Strcasestr}

\texttt{Strcasestr(}\textit{arg}, \textit{arg}\texttt{)}

This function may be used without a loaded font.

%%%%%%%%%%%%%%%%%%%%%%%%%%%%%%%%%%%%

\PEFuncRef{Strftime}

\texttt{Strftime(}\textit{format}, [\textit{isutc}, [\textit{locale}]]\texttt{)}

Generate a formatted time string, as the system \texttt{strftime()} function
would do.  If \textit{isutc} is present and 0, then the local time zone will
be used; otherwise, it will be UTC.  If \textit{locale} is specified as a
string, then that locale will be used; otherwise \texttt{strftime()} will be
called in the portable ``C'' locale.  In this respect FontAnvil differs from
from \FFdiff FontForge, which uses the current locale configured by
environment variables as the default.  The change is meant to ensure that
scripts will behave predictably on all systems unless they explicitly
require localization and handle it themselves.  This function may be used
without a loaded font.

%%%%%%%%%%%%%%%%%%%%%%%%%%%%%%%%%%%%

\PEFuncRef{Strlen}

\texttt{Strlen(}\textit{arg}\texttt{)}

This function may be used without a loaded font.

%%%%%%%%%%%%%%%%%%%%%%%%%%%%%%%%%%%%

\PEFuncRef{Strrstr}

\texttt{Strrstr(}\textit{arg}, \textit{arg}\texttt{)}

This function may be used without a loaded font.

%%%%%%%%%%%%%%%%%%%%%%%%%%%%%%%%%%%%

\PEFuncRef{Strskipint}

\texttt{Strskipint(}\textit{arg}, \textit{arg}\texttt{)}

This function may be used without a loaded font.

%%%%%%%%%%%%%%%%%%%%%%%%%%%%%%%%%%%%

\PEFuncRef{Strstr}

\texttt{Strstr(}\textit{arg}, \textit{arg}\texttt{)}

This function may be used without a loaded font.

%%%%%%%%%%%%%%%%%%%%%%%%%%%%%%%%%%%%

\PEFuncRef{Strsub}

\texttt{Strsub(}\textit{arg}, \textit{arg}, \textit{arg}\texttt{)}

This function may be used without a loaded font.

%%%%%%%%%%%%%%%%%%%%%%%%%%%%%%%%%%%%

\PEFuncRef{Strtod}

\texttt{Strtod(}\textit{arg}\texttt{)}

This function may be used without a loaded font.

%%%%%%%%%%%%%%%%%%%%%%%%%%%%%%%%%%%%

\PEFuncRef{Strtol}

\texttt{Strtol(}\textit{arg}, \textit{arg}\texttt{)}

This function may be used without a loaded font.

%%%%%%%%%%%%%%%%%%%%%%%%%%%%%%%%%%%%

\PEFuncRef{SubstitutionPoints}

\texttt{SubstitutionPoints(}\texttt{)}

%%%%%%%%%%%%%%%%%%%%%%%%%%%%%%%%%%%%

\PEFuncRef{Tan}

\texttt{Tan(}\textit{theta}\texttt{)}

Returns the tangent of \textit{theta}, which is measured in radians.
This function may be used without a loaded font.

%%%%%%%%%%%%%%%%%%%%%%%%%%%%%%%%%%%%

\PEFuncRef{ToLower}

\texttt{ToLower(}\textit{arg}\texttt{)}

This function may be used without a loaded font.

%%%%%%%%%%%%%%%%%%%%%%%%%%%%%%%%%%%%

\PEFuncRef{ToMirror}

\texttt{ToMirror(}\textit{arg}\texttt{)}

This function may be used without a loaded font.

%%%%%%%%%%%%%%%%%%%%%%%%%%%%%%%%%%%%

\PEFuncRef{ToString}

\texttt{ToString(}\textit{arg}\texttt{)}

This function may be used without a loaded font.

%%%%%%%%%%%%%%%%%%%%%%%%%%%%%%%%%%%%

\PEFuncRef{ToUpper}

\texttt{ToUpper(}\textit{arg}\texttt{)}

This function may be used without a loaded font.

%%%%%%%%%%%%%%%%%%%%%%%%%%%%%%%%%%%%

\PEFuncRef{Transform}

\texttt{Transform(}\textit{arg}, \textit{arg}, \textit{arg}, \textit{arg}, \textit{arg}, \textit{arg}\texttt{)}

%%%%%%%%%%%%%%%%%%%%%%%%%%%%%%%%%%%%

\PEFuncRef{TypeOf}

\texttt{TypeOf(}\textit{arg}\texttt{)}

This function may be used without a loaded font.

%%%%%%%%%%%%%%%%%%%%%%%%%%%%%%%%%%%%

\PEFuncRef{UCodePoint}

\texttt{UCodePoint(}\textit{int}\texttt{)}

Cast an integer to the special ``Unicode code point'' data type required by
some other scripting functions.  This function may be used without a loaded
font.

%%%%%%%%%%%%%%%%%%%%%%%%%%%%%%%%%%%%

\PEFuncRef{Ucs4}

\texttt{Ucs4(}\textit{str}\texttt{)}

Decode a UTF-8 string into an array of integers (not actually UCS4, despite
the name) expressing the code points
in the string.  Handling of illegal UTF-8, including encoded surrogate code
points, is undefined.  This is the inverse of the \texttt{Utf8()} function;
see also \texttt{Chr()} and \texttt{Ord()}, which operate on byte values
instead of code points.  This function may be used without a loaded font.

%%%%%%%%%%%%%%%%%%%%%%%%%%%%%%%%%%%%

\PEFuncRef{UnicodeAnnotationFromLib}

\texttt{UnicodeAnnotationFromLib(}\textit{arg}\texttt{)}

This function may be used without a loaded font.

%%%%%%%%%%%%%%%%%%%%%%%%%%%%%%%%%%%%

\PEFuncRef{UnicodeBlockEndFromLib}

\texttt{UnicodeBlockEndFromLib(}\textit{arg}\texttt{)}

This function may be used without a loaded font.

%%%%%%%%%%%%%%%%%%%%%%%%%%%%%%%%%%%%

\PEFuncRef{UnicodeBlockNameFromLib}

\texttt{UnicodeBlockNameFromLib(}\textit{arg}\texttt{)}

This function may be used without a loaded font.

%%%%%%%%%%%%%%%%%%%%%%%%%%%%%%%%%%%%

\PEFuncRef{UnicodeBlockStartFromLib}

\texttt{UnicodeBlockStartFromLib(}\textit{arg}\texttt{)}

This function may be used without a loaded font.

%%%%%%%%%%%%%%%%%%%%%%%%%%%%%%%%%%%%

\PEFuncRef{UnicodeFromName}

\texttt{UnicodeFromName(}\textit{arg}\texttt{)}

This function may be used without a loaded font.

%%%%%%%%%%%%%%%%%%%%%%%%%%%%%%%%%%%%

\PEFuncRef{UnicodeNameFromLib}

\texttt{UnicodeNameFromLib(}\textit{arg}\texttt{)}

This function may be used without a loaded font.

%%%%%%%%%%%%%%%%%%%%%%%%%%%%%%%%%%%%

\PEFuncRef{UnicodeNamesListVersion}

\texttt{UnicodeNamesListVersion(}\texttt{)}

This function may be used without a loaded font.

%%%%%%%%%%%%%%%%%%%%%%%%%%%%%%%%%%%%

\PEFuncRef{UnlinkReference}

\texttt{UnlinkReference(}\texttt{)}

%%%%%%%%%%%%%%%%%%%%%%%%%%%%%%%%%%%%

\PEFuncRef{Utf8}

\texttt{Utf8(}\textit{codes}\texttt{)}

Encode an array of integers in the range 0 to 0x10FFFF, or a such single
integer, into a UTF-8 string.  Surrogate code point values will result in
an illegal UTF-8 result, and zeroes will prematurely terminate the output. 
This is the inverse of the \texttt{Ucs4()} function; see also \texttt{Chr()}
and \texttt{Ord()}, which operate on byte values instead of code points. 
This function may be used without a loaded font.

%%%%%%%%%%%%%%%%%%%%%%%%%%%%%%%%%%%%

\PEFuncRef{VFlip}

\texttt{VFlip(}\textit{arg}\texttt{)}

%%%%%%%%%%%%%%%%%%%%%%%%%%%%%%%%%%%%

\PEFuncRef{VKernFromHKern}

\texttt{VKernFromHKern(}\texttt{)}

%%%%%%%%%%%%%%%%%%%%%%%%%%%%%%%%%%%%

\PEFuncRef{Validate}

\texttt{Validate(}[\textit{arg}]\texttt{)}

%%%%%%%%%%%%%%%%%%%%%%%%%%%%%%%%%%%%

\PEFuncRef{Wireframe}

\texttt{Wireframe(}\textit{arg}, \textit{arg}, \textit{arg}\texttt{)}

%%%%%%%%%%%%%%%%%%%%%%%%%%%%%%%%%%%%

\PEFuncRef{WorthOutputting}

\texttt{WorthOutputting(}\textit{arg}\texttt{)}

%%%%%%%%%%%%%%%%%%%%%%%%%%%%%%%%%%%%

\PEFuncRef{WriteStringToFile}

\texttt{WriteStringToFile(}\textit{arg}, \textit{arg}, \textit{arg}\texttt{)}

This function may be used without a loaded font.

%%%%%%%%%%%%%%%%%%%%%%%%%%%%%%%%%%%%

\PEFuncRef{WritePfm}

\texttt{WritePfm(}\textit{arg}\texttt{)}

%%%%%%%%%%%%%%%%%%%%%%%%%%%%%%%%%%%%%%%%%%%%%%%%%%%%%%%%%%%%%%%%%%%%%%%%

\section{Built-in functions in FontForge and not in FontAnvil}

\texttt{LoadPlugin()}\FFdiff\\
\texttt{LoadPluginDir()}

Removed because FontAnvil does not use plugins.

\noindent
\texttt{LoadPrefs()}\FFdiff\\
\texttt{SavePrefs()}

Removed because, to ensure consistent results from scripts, FontAnvil does
not store per-user ``preferences'' between runs.  For the moment, other
functions related to FontForge-style preference variables remain in the
language, but the values of these variables are initialized to defaults at
the start of each script run.

\noindent
\texttt{PrintFont()}\FFdiff\\
\texttt{PrintSetup()}

Removed because of the grossly disproportionate amount of unportable
interfacing code needed to talk to the native printing interfaces on each
operating system.  Some future version may support a more portable printing
feature, likely involving writing to files instead of interfacing directly
to the printer drivers.

\noindent
\texttt{Autotrace()}\FFdiff\\
\texttt{bAutoCounter()}\\
\texttt{bDontAutoHint()}\\
\texttt{bSubstitutionPoints()}\\
\texttt{BuildComposit()}\\
\texttt{GetPrefs()}

Misspellings of documented function names which FontForge once supported by
mistake and now retains for compatibility with hypothetical deployed
scripts that might have depended upon them.  No such deployed scripts are
actually known to exist.  FontAnvil does not retain the misspelled names.

\noindent
\texttt{AddATT()}\FFdiff\\
\texttt{DefaultATT()}\\
\texttt{PrivateToCvt()}\\
\texttt{RemoveATT()}\\
\texttt{SelectByATT()}

Deprecated functions that only produce error messages in FontForge; removed
entirely in FontAnvil.

%%%%%%%%%%%%%%%%%%%%%%%%%%%%%%%%%%%%%%%%%%%%%%%%%%%%%%%%%%%%%%%%%%%%%%%%

\section{Built-in variables}



%%%%%%%%%%%%%%%%%%%%%%%%%%%%%%%%%%%%%%%%%%%%%%%%%%%%%%%%%%%%%%%%%%%%%%%%

\iffalse
\chapter{SFD file format}

FIXME should have some more real information here instead of just
injunctions to the sinful.

\begin{framed}
I did not invent the SFD file format, and the person who did never fully
specified or documented it.  SFD is not a standardized format.  This
documentation is based partly on reverse engineering and it is descriptive,
not prescriptive.  That means if you create an SFD file with software other
than FontAnvil, you are not entitled to expect FontAnvil to read it,
\emph{not even if your file conforms to this documentation}; and that
applies to every piece of software that wrote the file as SFD,
not only the most recent one.  A problem was once observed in which
someone created a bad SFD file using homemade software, loaded it, saved
it, loaded it again, and it blew up after the second load because the data
really was invalid all along, but in a subtle way that survived
the first load/save pass.  The \emph{only}
supported way to create SFD files for FontAnvil to read is by writing them
originally with FontAnvil, using no data that ever passed through an SFD
file created by other software.  In particular, exchanging SFD files
between FontAnvil and FontForge, although I expect it will usually work,
is not supported.
\end{framed}

An SFD file is not a canonical form, in the sense of that term used by
mathematicians.  It is normal and expected that loading an SFD file and then
saving it may produce a file not byte-for-byte identical to the original. 
Loading and saving should identically preserve the  ``font,'' that is to
say the abstract entity consisting of glyph outlines associated to slot
numbers, named metadata strings, and so on; but many different SFD files can
realize any given font, and the only way to preserve a specific file
identically is to retain a copy outside FontAnvil.  This fact may have
implications for storing fonts in version control systems.  Note that few,
if any, other font file formats are canonical forms either.

FIXME

It should be obvious that the purpose of using UTF-7 encoding here is to
hide characters like newline (which would otherwise end the field) from the
parser, so it is absolutely necessary to encode such characters using the
Base64-style escape sequences of UTF-7 even if you think a string that left
these characters literal could also be syntactically valid UTF-7 under some
third-party definition of UTF-7.  Without the requirement for newlines to be
encoded there would be no way to distinguish a multi-line value that
happened to contain what looked like SFD field names, from a single-line
value followed by the rest of the file.  Exactly which characters must be
encoded for safe parsing is not defined beyond the general rule that
FontAnvil should read anything that FontAnvil wrote uninfluenced.  The
safest thing for other software to do is to put the entire field value in a
single Base64-style escape sequence whether it needs it or not.  (That is
what Tsukurimashou does.) Note that RFC-2152 does not require any specific
characters to ever be literal in UTF-7 (it at most \emph{allows} some
literal characters), and in any case RFC-2152 would not be authoritative
over the observed behaviour of the code.

The number of glyphs must meet the constraints of the font's encoding. 
FontAnvil checks, as does the latest FontForge; but some versions of
FontForge still in popular use do not, and may crash if this rule is not
observed.  See Section~\ref{sec:data-model} for more information about
encodings.

The order in which glyphs are stored in an SFD file is not significant, will
not in general be preserved through a load/save cycle, and will not affect
the order in which glyphs are stored when generating any non-SFD format,
such as TrueType.  Slot numbers, not the physical sequence of the records in
the file, define the significant ordering of glyphs in a font.

\fi
%%%%%%%%%%%%%%%%%%%%%%%%%%%%%%%%%%%%%%%%%%%%%%%%%%%%%%%%%%%%%%%%%%%%%%%%

\appendix

\onecolumn

\chapter{Licensing}

George Williams put most of his work on PfaEdit, and subsequently FontForge,
under licensing notices such as this one:

\begin{quotation}
Copyright \copyright\ [years] by George Williams

Redistribution and use in source and binary forms, with or without
modification, are permitted provided that the following conditions are met:

Redistributions of source code must retain the above copyright notice, this
list of conditions and the following disclaimer.

Redistributions in binary form must reproduce the above copyright notice,
this list of conditions and the following disclaimer in the documentation
and/or other materials provided with the distribution.

The name of the author may not be used to endorse or promote products
derived from this software without specific prior written permission.

THIS SOFTWARE IS PROVIDED BY THE AUTHOR ``AS IS'' AND ANY EXPRESS OR IMPLIED
WARRANTIES, INCLUDING, BUT NOT LIMITED TO, THE IMPLIED WARRANTIES OF
MERCHANTABILITY AND FITNESS FOR A PARTICULAR PURPOSE ARE DISCLAIMED.  IN NO
EVENT SHALL THE AUTHOR BE LIABLE FOR ANY DIRECT, INDIRECT, INCIDENTAL,
SPECIAL, EXEMPLARY, OR CONSEQUENTIAL DAMAGES (INCLUDING, BUT NOT LIMITED TO,
PROCUREMENT OF SUBSTITUTE GOODS OR SERVICES; LOSS OF USE, DATA, OR PROFITS;
OR BUSINESS INTERRUPTION) HOWEVER CAUSED AND ON ANY THEORY OF LIABILITY,
WHETHER IN CONTRACT, STRICT LIABILITY, OR TORT (INCLUDING NEGLIGENCE OR
OTHERWISE) ARISING IN ANY WAY OUT OF THE USE OF THIS SOFTWARE, EVEN IF
ADVISED OF THE POSSIBILITY OF SUCH DAMAGE.
\end{quotation}

That is what is commonly called a \emph{Three-Clause BSD License}.  There
were many subsequent contributors to the software (see the \texttt{AUTHORS}
file in the root of a distribution tarball or version control checkout) and
many of them were content to keep the same license terms in place, with or
without adding their own names and years to the copyright notice at the
top.

However, some contributors have placed additional restrictions on their
work, most notably \emph{GNU GPL3+} licenses like this one:

\begin{quotation}
Copyright \copyright\ [year] [contributor's name]

This program is free software: you can redistribute it and/or modify
it under the terms of the GNU General Public License as published by
the Free Software Foundation, either version 3 of the License, or
(at your option) any later version.

This program is distributed in the hope that it will be useful,
but WITHOUT ANY WARRANTY; without even the implied warranty of
MERCHANTABILITY or FITNESS FOR A PARTICULAR PURPOSE.  See the
GNU General Public License for more details.

You should have received a copy of the GNU General Public License
along with this program.  If not, see
\url{http://www.gnu.org/licenses/}.
\end{quotation}

The Free Software Foundation takes the position that the three-clause BSD
license is GPL-compatible,%
\footnote{\url{http://www.gnu.org/licenses/license-list.html\#ModifiedBSD}}
meaning that it is legally permissible for a package that is under GPL3+ as
a whole to include material that is under three-clause BSD.  The presence of
GPL3+ contributions, however, forces the package as a whole to be licensed
GPL3+.  That being the case, both FontForge and FontAnvil should be treated
as GPL3+, just with the awareness that some files may also be used
separately from the package under the less restrictive three-clause BSD
license.

It is the current practice of the FontForge project to encourage
contributors of new material to apply GPL3+ notices to any new files, but
retain the BSD notices on files that already have those.  There was an
incident in which someone tried to apply a patch someone else had written to
a currently BSD-licensed file in FontForge---with the patch to become
roughly 1/1000th of the file's total volume.  The author of the patch
demanded that the whole file should become GPL3+, overriding the BSD notice
on it and the apparent intentions of the previous contributors to that file. 
I would like to avoid such incidents.

A few files have other distribution terms.  In particular, some parts of the
build system have very permissive licenses.

FontForge attempts to maintain a list of all the licensing terms of all the
files in the project; but their list has never been up to date, cannot
reasonably be expected to ever be up to date nor to stay up to date even if
it ever is at one moment, currently contains incorrect information, and
seems unnecessary.  I do not propose to make such a list for FontAnvil.

The current licensing policy for FontAnvil is substantially the same as that
of FontForge:
\begin{itemize}
\item FontAnvil as a whole is covered by the GPL version 3, or any later
version.
\item Some files in FontAnvil are also available under less restrictive
licenses.  You must consult the notices in those files for details.
\item I will place GPL3+ notices on new files I create within FontAnvil,
and encourage others to do the same.
\item I will leave files with existing broader-than-GPL notices under their
existing notices (possibly adding my own name and year copyright lines), and
encourage others to do the same.
\item I will not accept contributions that entail drastic changes to the
licensing status of work done by persons other than the contributor, and I
will discourage the submission of such contributions.
\end{itemize}

Finally, note that although FontAnvil is associated with the Tsukurimashou
Project, its licensing is not identical to that of other things included in
the Tsukurimashou Project.

%%%%%%%%%%%%%%%%%%%%%%%%%%%%%%%%%%%%%%%%%%%%%%%%%%%%%%%%%%%%%%%%%%%%%%%%

\twocolumn

\clearpage
\phantomsection\currentpdfbookmark{Index}{bkm:index}
\printindex

%%%%%%%%%%%%%%%%%%%%%%%%%%%%%%%%%%%%%%%%%%%%%%%%%%%%%%%%%%%%%%%%%%%%%%%%

\end{document}
