\chapter{Language reference}

\begin{framed}
I did not invent the PE scripting language, and the person who did never
fully specified or documented it.  This documentation is based partly on
reverse engineering; is descriptive, not prescriptive; and may not be
complete, nor even correct as far as it goes.  The only way to be sure
what a PE script will really do is to run it and find out, like
Rikki-Tikki-Tavi.
\end{framed}

\section{Basic syntax}

Scripts are text files.  The traditional filename extension is \texttt{.pe}~;
scripts in the wild have also been seen using a \texttt{.ff} extension.

Comments may be marked in any of these ways:
\begin{verbatim}
# hash for a shell-like comment to the end of the line

// two slashes for a C++-like comment to the end of the line

/*
C-style comment delimiters,
which may cover multiple lines.
*/
\end{verbatim}

FontForge looks for a shebang line at the top, pointing at itself, and may
also attempt to recognize command-line options there to distinguish between
PE scripts and Python scripts.  FontAnvil only supports PE scripts and is
planned to more or less ignore the shebang; however, this code has not yet
been worked over since its importation from FontForge and may not really
work as desired at the moment.

\begin{framed}
Newlines are syntactically significant, marking the ends of
statements.
To continue a statement onto more than one line, you must use a backslash to
escape the newline.  
\end{framed}

Semicolons also mark the ends of statements, and may be used to join
multiple statements onto a single line.  Semicolons at the ends of lines
create empty statements, which are ignored.

\begin{framed}
PE script is case sensitive for reserved words, variable names,
and built-in function names.
\end{framed}

\section{Data types, variables, and scope}

Values have associated types.  Variables can hold values of arbitrary type
and remember what type they are.  The types are:
\begin{itemize}
\item integer
\item floating-point number
\item Unicode code point (note that this is a distinct data type from
``integer'')
\item string
\item array
\end{itemize}

Syntax for constant values looks like this:
\begin{verbatim}
# integers in decimal, hexadecimal, or octal, using C syntax
123     # first digit nonzero means decimal
0x52    # first digits 0x means hex, this is 82 decimal
041     # first digit zero and not hex means octal, this is 33 decimal

# floating-point numbers indicated by the decimal point; note the decimal
# point is always . regardless of locale
123.45  # basic decimal float
4.9e5   # scientific notation, this is 490000

# Unicode code points are hexadecimal numbers marked by 0u
0u1f4a9 # everybody's favourite

# strings have single or double quotes
'Single'
"double"
"foo\nbar" # \n for newline, in both kinds of strings
"foo\
bar" # backslash continues string past a line break, this is "foobar"
"foo\\bar" # backslash escapes other characters, this is "foo\bar" 
'"' # a string normally ends with the same quote that opened it

# unescaped newline also validly ends a string, but don't do this!
# current FontAnvil supports it for FontForge compatibility
# some future version may make it an error
"foo

# array literals use square brackets and commas
[1,2,3,0uABC,'foo']
\end{verbatim}

\begin{framed}
Literal string constants in PE script syntax are limited to 256 characters. 
You can, however, construct longer strings with multiple literals and the
concatenation operator.
\end{framed}

\begin{framed}
The language seems intended to allow arrays to have more than one dimension
(i.e.\ each element of an array may itself be an array) but such arrays are
currently broken in both FontForge and FontAnvil, and usually cause the
interpreter to crash.  I hope to fix this bug in FontAnvil, but if I fix it
and FontForge doesn't, then any scripts that make use of multidimensional
arrays will be incompatible with FontForge.
\end{framed}

\section{Operators}

FIXME

\section{Control structures}

FIXME

\section{Most of the built-in functions}

FIXME

This section should document the functions.

%%%%%%%%%%%%%%%%%%%%%%%%%%%%%%%%%%%%%%%%%%%%%%%%%%%%%%%%%%%%%%%%%%%%%%%%

\PEFuncRef{ATan2}

\noindent\texttt{ATan2(}\textit{y}, \textit{x}\texttt{)}

Returns the arctangent of $y/x$ in radians, using the signs of the arguments
to choose the quadrant.  \emph{Note the order of the arguments, with $y$
first.}  This function mimics the behaviour of the C math library
\texttt{atan2()} function.  This function may be used without a
loaded font.

\PEFuncRef{AddATT}

\noindent\texttt{AddATT(}, \ldots\texttt{)}

\PEFuncRef{AddAccent}

\noindent\texttt{AddAccent(}\textit{arg}, \textit{arg}\texttt{)}

\PEFuncRef{AddAnchorClass}

\noindent\texttt{AddAnchorClass(}\textit{arg}, \textit{arg}, \textit{arg}\texttt{)}

\PEFuncRef{AddAnchorPoint}

\noindent\texttt{AddAnchorPoint(}\textit{arg}, \textit{arg}, \textit{arg}, \textit{arg}, \textit{arg}\texttt{)}

\PEFuncRef{AddDHint}

\noindent\texttt{AddDHint(}\textit{arg}, \textit{arg}, \textit{arg}, \textit{arg}, \textit{arg}, \textit{arg}\texttt{)}

\PEFuncRef{AddExtrema}

\noindent\texttt{AddExtrema(}\textit{arg}\texttt{)}

\PEFuncRef{AddHHint}

\noindent\texttt{AddHHint(}, \ldots\texttt{)}

\PEFuncRef{AddInstrs}

\noindent\texttt{AddInstrs(}\textit{arg}, \textit{arg}, \textit{arg}\texttt{)}

\PEFuncRef{AddLookup}

\noindent\texttt{AddLookup(}\textit{arg}, \textit{arg}, \textit{arg}, \textit{arg}, \textit{arg}\texttt{)}

\PEFuncRef{AddLookupSubtable}

\noindent\texttt{AddLookupSubtable(}\textit{arg}, \textit{arg}, \textit{arg}\texttt{)}

\PEFuncRef{AddPosSub}

\noindent\texttt{AddPosSub(}\textit{arg}, \textit{arg}, \textit{arg}, [\textit{arg}], [\textit{arg}], [\textit{arg}], [\textit{arg}], [\textit{arg}], [\textit{arg}], [\textit{arg}]\texttt{)}

\PEFuncRef{AddSizeFeature}

\noindent\texttt{AddSizeFeature(}\textit{arg}, \textit{arg}, [\textit{arg}], [\textit{arg}], [\textit{arg}]\texttt{)}

\PEFuncRef{AddVHint}

\noindent\texttt{AddVHint(}, \ldots\texttt{)}

\PEFuncRef{ApplySubstitution}

\noindent\texttt{ApplySubstitution(}\textit{arg}, \textit{arg}, \textit{arg}\texttt{)}

\PEFuncRef{Array}

\noindent\texttt{Array(}\textit{size}\texttt{)}

Creates and returns a value of array type, with the given number of
elements initialized to void.
This function may be used without a loaded font.

\PEFuncRef{AskUser}

\noindent\texttt{AskUser(}\textit{prompt}, [\textit{default}]\texttt{)}

Asks the user a question.  The string \textit{prompt} is written out to
standard output, and the next line from standard input is captured to become
the return value.  A line is terminated by a newline character, which will
be included in the return value.  On error, or if the user enters a blank
line, the return value will be \textit{default} if specified, or an empty
string if not.
This function may be used without a loaded font.

\PEFuncRef{AutoCounter}

\noindent\texttt{AutoCounter(}\texttt{)}

\PEFuncRef{AutoHint}

\noindent\texttt{AutoHint(}\texttt{)}

\PEFuncRef{AutoInstr}

\noindent\texttt{AutoInstr(}\texttt{)}

\PEFuncRef{AutoKern}

\noindent\texttt{AutoKern(}\textit{arg}, \textit{arg}, \textit{arg}, \textit{arg}\texttt{)}

\PEFuncRef{AutoTrace}

\noindent\texttt{AutoTrace(}, \ldots\texttt{)}

\PEFuncRef{AutoWidth}

\noindent\texttt{AutoWidth(}\textit{arg}, \textit{arg}, [\textit{arg}]\texttt{)}

\PEFuncRef{Autotrace}

\noindent\texttt{Autotrace(}\texttt{)}

\PEFuncRef{BitmapsAvail}

\noindent\texttt{BitmapsAvail(}, \ldots\texttt{)}

\PEFuncRef{BitmapsRegen}

\noindent\texttt{BitmapsRegen(}, \ldots\texttt{)}

\PEFuncRef{BuildAccented}

\noindent\texttt{BuildAccented(}\texttt{)}

\PEFuncRef{BuildComposite}

\noindent\texttt{BuildComposite(}\texttt{)}

\PEFuncRef{BuildDuplicate}

\noindent\texttt{BuildDuplicate(}\texttt{)}

\PEFuncRef{CIDChangeSubFont}

\noindent\texttt{CIDChangeSubFont(}\textit{arg}\texttt{)}

\PEFuncRef{CIDFlatten}

\noindent\texttt{CIDFlatten(}\texttt{)}

\PEFuncRef{CIDFlattenByCMap}

\noindent\texttt{CIDFlattenByCMap(}\textit{arg}\texttt{)}

\PEFuncRef{CIDSetFontNames}

\noindent\texttt{CIDSetFontNames(}\textit{arg}, \textit{arg}, [\textit{arg}], [\textit{arg}], [\textit{arg}], [\textit{arg}]\texttt{)}

\PEFuncRef{CanonicalContours}

\noindent\texttt{CanonicalContours(}\texttt{)}

\PEFuncRef{CanonicalStart}

\noindent\texttt{CanonicalStart(}\texttt{)}

\PEFuncRef{Ceil}

\noindent\texttt{Ceil(}\textit{x}\texttt{)}

Returns $\lceil x \rceil$.  That is the ceiling of $x$, or the least
integer greater than or equal to $x$.  This function may be used without a
loaded font.

\PEFuncRef{CenterInWidth}

\noindent\texttt{CenterInWidth(}\texttt{)}

\PEFuncRef{ChangePrivateEntry}

\noindent\texttt{ChangePrivateEntry(}\textit{arg}, \textit{arg}\texttt{)}

\PEFuncRef{ChangeWeight}

\noindent\texttt{ChangeWeight(}[\textit{arg}]\texttt{)}

\PEFuncRef{CharCnt}

\noindent\texttt{CharCnt()}

Return the number of glyph slots in the current font, including both encoded
and unencoded slots.

\PEFuncRef{CharInfo}

\noindent\texttt{CharInfo(}\textit{arg}, \textit{arg}\texttt{)}

\PEFuncRef{CheckForAnchorClass}

\noindent\texttt{CheckForAnchorClass(}\textit{arg}\texttt{)}

\PEFuncRef{Chr}

\noindent\texttt{Chr(}\textit{int}\texttt{)}

Takes a single integer in the range -128 to 255 and returns a string
containing that byte, folding negative values into two's complement
notation.

\noindent\texttt{Chr(}\textit{array}\texttt{)}

Takes an array of integers in the range -128 to 255 and returns a string
consisting of those bytes.

This function may be used without a loaded font.  Note that the integers are
\emph{byte} values.  Code points U+0080 and beyond may be constructed by
spelling them out in UTF-8.  It is also possible, but not recommended, to
construct strings that are invalid UTF-8.

I have submitted a patch to FontForge, but as of the current writing (June
2015), FontForge does not document its support of negative numbers, and
documents but does not correctly implement array-valued
arguments.  FontAnvil behaves as documented here.

\PEFuncRef{Clear}

\noindent\texttt{Clear()}

\PEFuncRef{ClearBackground}

\noindent\texttt{ClearBackground(}\texttt{)}

\PEFuncRef{ClearCharCounterMasks}

\noindent\texttt{ClearCharCounterMasks(}\texttt{)}

\PEFuncRef{ClearGlyphCounterMasks}

\noindent\texttt{ClearGlyphCounterMasks(}\texttt{)}

\PEFuncRef{ClearHints}

\noindent\texttt{ClearHints(}[\textit{arg}]\texttt{)}

\PEFuncRef{ClearInstrs}

\noindent\texttt{ClearInstrs(}\texttt{)}

\PEFuncRef{ClearPrivateEntry}

\noindent\texttt{ClearPrivateEntry(}\textit{arg}\texttt{)}

\PEFuncRef{ClearTable}

\noindent\texttt{ClearTable(}\textit{arg}\texttt{)}

\PEFuncRef{Close}

\noindent\texttt{Close(}\texttt{)}

\PEFuncRef{CompareFonts}

\noindent\texttt{CompareFonts(}\textit{arg}, \textit{arg}, \textit{arg}\texttt{)}

\PEFuncRef{CompareGlyphs}

\noindent\texttt{CompareGlyphs(}[\textit{arg}], [\textit{arg}], [\textit{arg}], [\textit{arg}], [\textit{arg}], [\textit{arg}]\texttt{)}

\PEFuncRef{ControlAfmLigatureOutput}

\noindent\texttt{ControlAfmLigatureOutput(}\textit{arg}, \textit{arg}\texttt{)}

\PEFuncRef{ConvertByCMap}

\noindent\texttt{ConvertByCMap(}\textit{arg}\texttt{)}

\PEFuncRef{ConvertToCID}

\noindent\texttt{ConvertToCID(}\textit{arg}, \textit{arg}, \textit{arg}\texttt{)}

\PEFuncRef{Copy}

\noindent\texttt{Copy(}\texttt{)}

\PEFuncRef{CopyAnchors}

\noindent\texttt{CopyAnchors(}\texttt{)}

\PEFuncRef{CopyFgToBg}

\noindent\texttt{CopyFgToBg(}\texttt{)}

\PEFuncRef{CopyGlyphFeatures}

\noindent\texttt{CopyGlyphFeatures(}, \ldots\texttt{)}

\PEFuncRef{CopyLBearing}

\noindent\texttt{CopyLBearing(}\texttt{)}

\PEFuncRef{CopyRBearing}

\noindent\texttt{CopyRBearing(}\texttt{)}

\PEFuncRef{CopyReference}

\noindent\texttt{CopyReference(}\texttt{)}

\PEFuncRef{CopyUnlinked}

\noindent\texttt{CopyUnlinked(}\texttt{)}

\PEFuncRef{CopyVWidth}

\noindent\texttt{CopyVWidth(}\texttt{)}

\PEFuncRef{CopyWidth}

\noindent\texttt{CopyWidth(}\texttt{)}

\PEFuncRef{CorrectDirection}

\noindent\texttt{CorrectDirection(}\textit{arg}\texttt{)}

\PEFuncRef{Cos}

\noindent\texttt{Cos(}\textit{theta}\texttt{)}

Returns the cosine of \textit{theta}, which is measured in radians.
This function may be used without a loaded font.

\PEFuncRef{Cut}

\noindent\texttt{Cut(}\texttt{)}

\PEFuncRef{DebugCrashFontForge}

\noindent\texttt{DebugCrashFontForge(}\ldots\texttt{)}

Causes FontAnvil [sic] to attempt to write to a null pointer, which should
crash the interpreter.  This may be of some use in debugging.  Arguments are
ignored.  Function name retained for compatibility.  In FontForge, this
function requires a loaded font, but that limitation is removed in FontAnvil.

\PEFuncRef{DefaultATT}

\noindent\texttt{DefaultATT(}, \ldots\texttt{)}

\PEFuncRef{DefaultOtherSubrs}

\noindent\texttt{DefaultOtherSubrs(}\texttt{)}

This function may be used without a loaded font.

\PEFuncRef{DefaultRoundToGrid}

\noindent\texttt{DefaultRoundToGrid(}\texttt{)}

\PEFuncRef{DefaultUseMyMetrics}

\noindent\texttt{DefaultUseMyMetrics(}\texttt{)}

\PEFuncRef{DetachAndRemoveGlyphs}

\noindent\texttt{DetachAndRemoveGlyphs(}, \ldots\texttt{)}

\PEFuncRef{DetachGlyphs}

\noindent\texttt{DetachGlyphs(}, \ldots\texttt{)}

\PEFuncRef{DontAutoHint}

\noindent\texttt{DontAutoHint(}\texttt{)}

\PEFuncRef{DrawsSomething}

\noindent\texttt{DrawsSomething(}\textit{arg}\texttt{)}

\PEFuncRef{Error}

\noindent\texttt{Error(}\textit{msg}\texttt{)}

This function may be used without a loaded font.

\PEFuncRef{Exp}

\noindent\texttt{Exp(}\textit{x}\texttt{)}

Compute the exponential function, $e^x$.
This function may be used without a loaded font.

\PEFuncRef{ExpandStroke}

\noindent\texttt{ExpandStroke(}\textit{arg}, \textit{arg}, [\textit{arg}], [\textit{arg}], [\textit{arg}], [\textit{arg}]\texttt{)}

\PEFuncRef{Export}

\noindent\texttt{Export(}\textit{arg}, \textit{arg}\texttt{)}

\PEFuncRef{FileAccess}

\noindent\texttt{FileAccess(}\textit{arg}, \textit{arg}\texttt{)}

This function may be used without a loaded font.

\PEFuncRef{FindIntersections}

\noindent\texttt{FindIntersections(}\texttt{)}

\PEFuncRef{FindOrAddCvtIndex}

\noindent\texttt{FindOrAddCvtIndex(}\textit{arg}, \textit{arg}\texttt{)}

\PEFuncRef{Floor}

\noindent\texttt{Floor(}\textit{arg}\texttt{)}

Returns $\lfloor x \rfloor$.  That is the floor of $x$, or the greatest
integer less than or equal to $x$.  This function may be used without a
loaded font.

\PEFuncRef{FontImage}

\noindent\texttt{FontImage(}\textit{arg}, \textit{arg}, \textit{arg}, [\textit{arg}]\texttt{)}

\PEFuncRef{FontsInFile}

\noindent\texttt{FontsInFile(}\textit{arg}\texttt{)}

This function may be used without a loaded font.

\PEFuncRef{Generate}

\noindent\texttt{Generate(}\textit{arg}, \textit{arg}, [\textit{arg}], [\textit{arg}], [\textit{arg}], [\textit{arg}]\texttt{)}

\PEFuncRef{GenerateFamily}

\noindent\texttt{GenerateFamily(}\textit{arg}, \textit{arg}, \textit{arg}, \textit{arg}\texttt{)}

\PEFuncRef{GenerateFeatureFile}

\noindent\texttt{GenerateFeatureFile(}\textit{arg}, \textit{arg}\texttt{)}

\PEFuncRef{GetAnchorPoints}

\noindent\texttt{GetAnchorPoints(}, \ldots\texttt{)}

\PEFuncRef{GetCoverageCounts}

\noindent\texttt{GetCoverageCounts()}

Print (to standard output) a table listing all the built-in functions in the
interpreter and how many times each one has been called in the current run
of FontAnvil; this information may be useful in verifying the coverage of an
interpreter test suite.
This function may be used without a loaded font.  This function is a
FontAnvil extension and is not available in FontForge.  The details of its
operation may change in some future version.

\PEFuncRef{GetCvtAt}

\noindent\texttt{GetCvtAt(}\textit{arg}\texttt{)}

\PEFuncRef{GetEnv}

\noindent\texttt{GetEnv(}\textit{arg}\texttt{)}

This function may be used without a loaded font.

\PEFuncRef{GetFontBoundingBox}

\noindent\texttt{GetFontBoundingBox(}\texttt{)}

\PEFuncRef{GetLookupInfo}

\noindent\texttt{GetLookupInfo(}\textit{arg}\texttt{)}

\PEFuncRef{GetLookupOfSubtable}

\noindent\texttt{GetLookupOfSubtable(}\textit{arg}\texttt{)}

\PEFuncRef{GetLookupSubtables}

\noindent\texttt{GetLookupSubtables(}\textit{arg}\texttt{)}

\PEFuncRef{GetLookups}

\noindent\texttt{GetLookups(}\textit{arg}\texttt{)}

\PEFuncRef{GetMaxpValue}

\noindent\texttt{GetMaxpValue(}\textit{arg}\texttt{)}

\PEFuncRef{GetOS2Value}

\noindent\texttt{GetOS2Value(}, \ldots\texttt{)}

\PEFuncRef{GetPosSub}

\noindent\texttt{GetPosSub(}\textit{arg}\texttt{)}

\PEFuncRef{GetPref}

\noindent\texttt{GetPref(}\textit{arg}\texttt{)}

This function may be used without a loaded font.

\PEFuncRef{GetPrivateEntry}

\noindent\texttt{GetPrivateEntry(}\textit{arg}\texttt{)}

\PEFuncRef{GetSubtableOfAnchorClass}

\noindent\texttt{GetSubtableOfAnchorClass(}\textit{arg}\texttt{)}

\PEFuncRef{GetTTFName}

\noindent\texttt{GetTTFName(}\textit{arg}, \textit{arg}\texttt{)}

\PEFuncRef{GetTeXParam}

\noindent\texttt{GetTeXParam(}\textit{arg}\texttt{)}

\PEFuncRef{GlyphInfo}

\noindent\texttt{GlyphInfo(}, \ldots\texttt{)}

\PEFuncRef{HFlip}

\noindent\texttt{HFlip(}\textit{arg}\texttt{)}

\PEFuncRef{HasPreservedTable}

\noindent\texttt{HasPreservedTable(}\textit{arg}\texttt{)}

\PEFuncRef{HasPrivateEntry}

\noindent\texttt{HasPrivateEntry(}\textit{arg}\texttt{)}

\PEFuncRef{Import}

\noindent\texttt{Import(}\textit{arg}, \textit{arg}, [\textit{arg}]\texttt{)}

\PEFuncRef{InFont}

\noindent\texttt{InFont(}[\textit{arg}]\texttt{)}

\PEFuncRef{Inline}

\noindent\texttt{Inline(}\textit{arg}, \textit{arg}\texttt{)}

\PEFuncRef{Int}

\noindent\texttt{Int(}\textit{x}\texttt{)}

Converts a real number or Unicode code point value to an integer, by means
of the C compiler's type coercion.  In the case of real numbers, that means
$x$ will be rounded toward zero.  An integer argument will be returned
unchanged.
This function may be used without a loaded font.

\PEFuncRef{InterpolateFonts}

\noindent\texttt{InterpolateFonts(}\textit{arg}, \textit{arg}, \textit{arg}\texttt{)}

\PEFuncRef{IsAlNum}

\noindent\texttt{IsAlNum(}\textit{arg}\texttt{)}

This function may be used without a loaded font.

\PEFuncRef{IsAlpha}

\noindent\texttt{IsAlpha(}\textit{arg}\texttt{)}

This function may be used without a loaded font.

\PEFuncRef{IsDigit}

\noindent\texttt{IsDigit(}\textit{arg}\texttt{)}

This function may be used without a loaded font.

\PEFuncRef{IsFinite}

\noindent\texttt{IsFinite(}\textit{arg}\texttt{)}

This function may be used without a loaded font.

\PEFuncRef{IsHexDigit}

\noindent\texttt{IsHexDigit(}\textit{arg}\texttt{)}

This function may be used without a loaded font.

\PEFuncRef{IsLower}

\noindent\texttt{IsLower(}\textit{arg}\texttt{)}

This function may be used without a loaded font.

\PEFuncRef{IsNan}

\noindent\texttt{IsNan(}\textit{arg}\texttt{)}

This function may be used without a loaded font.

\PEFuncRef{IsSpace}

\noindent\texttt{IsSpace(}\textit{arg}\texttt{)}

This function may be used without a loaded font.

\PEFuncRef{IsUpper}

\noindent\texttt{IsUpper(}\textit{arg}\texttt{)}

This function may be used without a loaded font.

\PEFuncRef{Italic}

\noindent\texttt{Italic(}[\textit{arg}], [\textit{arg}], [\textit{arg}], [\textit{arg}], [\textit{arg}], [\textit{arg}], [\textit{arg}], [\textit{arg}], [\textit{arg}]\texttt{)}

\PEFuncRef{Join}

\noindent\texttt{Join(}\texttt{)}

\PEFuncRef{LoadEncodingFile}

\noindent\texttt{LoadEncodingFile(}\textit{arg}, \textit{arg}\texttt{)}

This function may be used without a loaded font.

\PEFuncRef{LoadNamelist}

\noindent\texttt{LoadNamelist(}\textit{arg}\texttt{)}

This function may be used without a loaded font.

\PEFuncRef{LoadNamelistDir}

\noindent\texttt{LoadNamelistDir(}[\textit{arg}]\texttt{)}

This function may be used without a loaded font.

\PEFuncRef{LoadStringFromFile}

\noindent\texttt{LoadStringFromFile(}\textit{arg}\texttt{)}

This function may be used without a loaded font.

\PEFuncRef{LoadTableFromFile}

\noindent\texttt{LoadTableFromFile(}\textit{arg}, \textit{arg}\texttt{)}

\PEFuncRef{Log}

\noindent\texttt{Log(}\textit{x}\texttt{)}

Returns $\ln x$, that is the natural (base $e$) logarithm.  This is an error
if $x$ is zero, negative, or not a number.
This function may be used without a loaded font.

\PEFuncRef{LookupStoreLigatureInAfm}

\noindent\texttt{LookupStoreLigatureInAfm(}\textit{arg}, \textit{arg}\texttt{)}

\PEFuncRef{MMAxisBounds}

\noindent\texttt{MMAxisBounds(}\textit{arg}\texttt{)}

\PEFuncRef{MMAxisNames}

\noindent\texttt{MMAxisNames(}\texttt{)}

\PEFuncRef{MMBlendToNewFont}

\noindent\texttt{MMBlendToNewFont(}, \ldots\texttt{)}

\PEFuncRef{MMChangeInstance}

\noindent\texttt{MMChangeInstance(}\textit{arg}\texttt{)}

\PEFuncRef{MMChangeWeight}

\noindent\texttt{MMChangeWeight(}, \ldots\texttt{)}

\PEFuncRef{MMInstanceNames}

\noindent\texttt{MMInstanceNames(}\texttt{)}

\PEFuncRef{MMWeightedName}

\noindent\texttt{MMWeightedName(}\texttt{)}

\PEFuncRef{MakeLine}

\noindent\texttt{MakeLine(}, \ldots\texttt{)}

\PEFuncRef{MergeFeature}

\noindent\texttt{MergeFeature(}\textit{arg}\texttt{)}

\PEFuncRef{MergeFonts}

\noindent\texttt{MergeFonts(}\textit{arg}, \textit{arg}\texttt{)}

\PEFuncRef{MergeKern}

\noindent\texttt{MergeKern(}\textit{arg}\texttt{)}

\PEFuncRef{MergeLookupSubtables}

\noindent\texttt{MergeLookupSubtables(}\textit{arg}, \textit{arg}\texttt{)}

\PEFuncRef{MergeLookups}

\noindent\texttt{MergeLookups(}\textit{arg}, \textit{arg}\texttt{)}

\PEFuncRef{Move}

\noindent\texttt{Move(}\textit{arg}, \textit{arg}\texttt{)}

\PEFuncRef{MoveReference}

\noindent\texttt{MoveReference(}\textit{arg}, \textit{arg}, \textit{arg}, \ldots\texttt{)}

\PEFuncRef{MultipleEncodingsToReferences}

\noindent\texttt{MultipleEncodingsToReferences(}\texttt{)}

\PEFuncRef{NameFromUnicode}

\noindent\texttt{NameFromUnicode(}\textit{arg}, \textit{arg}\texttt{)}

This function may be used without a loaded font.

\PEFuncRef{NearlyHvCps}

\noindent\texttt{NearlyHvCps(}[\textit{arg}], [\textit{arg}]\texttt{)}

\PEFuncRef{NearlyHvLines}

\noindent\texttt{NearlyHvLines(}[\textit{arg}]\texttt{)}

\PEFuncRef{NearlyLines}

\noindent\texttt{NearlyLines(}[\textit{arg}]\texttt{)}

\PEFuncRef{New}

\noindent\texttt{New(}\texttt{)}

This function may be used without a loaded font.

\PEFuncRef{NonLinearTransform}

\noindent\texttt{NonLinearTransform(}\textit{arg}, \textit{arg}\texttt{)}

\PEFuncRef{Open}

\noindent\texttt{Open(}\textit{arg}, \textit{arg}\texttt{)}

This function may be used without a loaded font.

\PEFuncRef{Ord}

\noindent\texttt{Ord(}\textit{arg}, \textit{arg}\texttt{)}

This function may be used without a loaded font.

\PEFuncRef{Outline}

\noindent\texttt{Outline(}\textit{arg}\texttt{)}

\PEFuncRef{OverlapIntersect}

\noindent\texttt{OverlapIntersect(}\texttt{)}

\PEFuncRef{Paste}

\noindent\texttt{Paste(}\texttt{)}

\PEFuncRef{PasteInto}

\noindent\texttt{PasteInto(}\texttt{)}

\PEFuncRef{PasteWithOffset}

\noindent\texttt{PasteWithOffset(}\textit{arg}, \textit{arg}\texttt{)}

\PEFuncRef{PositionReference}

\noindent\texttt{PositionReference(}\textit{arg}, \textit{arg}, \textit{arg}, \ldots\texttt{)}

\PEFuncRef{PostNotice}

\noindent\texttt{PostNotice(}\textit{arg}\texttt{)}

This function may be used without a loaded font.

\PEFuncRef{Pow}

\noindent\texttt{Pow(}\textit{arg}, \textit{arg}\texttt{)}

This function may be used without a loaded font.

\PEFuncRef{PreloadCidmap}

\noindent\texttt{PreloadCidmap(}\textit{arg}, \textit{arg}, \textit{arg}, \textit{arg}\texttt{)}

This function may be used without a loaded font.

\PEFuncRef{Print}

\noindent\texttt{Print(}, \ldots\texttt{)}

This function may be used without a loaded font.

\PEFuncRef{PrintFont}

\noindent\texttt{PrintFont(}\textit{arg}, \textit{arg}, [\textit{arg}], [\textit{arg}]\texttt{)}

\PEFuncRef{PrintSetup}

\noindent\texttt{PrintSetup(}\textit{arg}, \textit{arg}, [\textit{arg}], [\textit{arg}]\texttt{)}

This function may be used without a loaded font.

\PEFuncRef{PrivateGuess}

\noindent\texttt{PrivateGuess(}\textit{arg}\texttt{)}

\PEFuncRef{Quit}

\noindent\texttt{Quit(}[\textit{arg}]\texttt{)}

This function may be used without a loaded font.

\PEFuncRef{Rand}

\noindent\texttt{Rand(}\texttt{)}

This function may be used without a loaded font.

\PEFuncRef{ReadOtherSubrsFile}

\noindent\texttt{ReadOtherSubrsFile(}\textit{arg}\texttt{)}

This function may be used without a loaded font.

\PEFuncRef{Real}

\noindent\texttt{Real(}\textit{arg}\texttt{)}

This function may be used without a loaded font.

\PEFuncRef{Reencode}

\noindent\texttt{Reencode(}\textit{arg}, \textit{arg}\texttt{)}

\PEFuncRef{RemoveAllKerns}

\noindent\texttt{RemoveAllKerns(}\texttt{)}

\PEFuncRef{RemoveAllVKerns}

\noindent\texttt{RemoveAllVKerns(}\texttt{)}

\PEFuncRef{RemoveAnchorClass}

\noindent\texttt{RemoveAnchorClass(}\textit{arg}\texttt{)}

\PEFuncRef{RemoveDetachedGlyphs}

\noindent\texttt{RemoveDetachedGlyphs(}, \ldots\texttt{)}

\PEFuncRef{RemoveLookup}

\noindent\texttt{RemoveLookup(}\textit{arg}, \textit{arg}\texttt{)}

\PEFuncRef{RemoveLookupSubtable}

\noindent\texttt{RemoveLookupSubtable(}\textit{arg}, \textit{arg}\texttt{)}

\PEFuncRef{RemoveOverlap}

\noindent\texttt{RemoveOverlap(}\texttt{)}

\PEFuncRef{RemovePosSub}

\noindent\texttt{RemovePosSub(}\textit{arg}\texttt{)}

\PEFuncRef{RemovePreservedTable}

\noindent\texttt{RemovePreservedTable(}\textit{arg}\texttt{)}

\PEFuncRef{RenameGlyphs}

\noindent\texttt{RenameGlyphs(}\textit{arg}\texttt{)}

\PEFuncRef{ReplaceCharCounterMasks}

\noindent\texttt{ReplaceCharCounterMasks(}\textit{arg}\texttt{)}

\PEFuncRef{ReplaceCvtAt}

\noindent\texttt{ReplaceCvtAt(}\textit{arg}, \textit{arg}\texttt{)}

\PEFuncRef{ReplaceGlyphCounterMasks}

\noindent\texttt{ReplaceGlyphCounterMasks(}\textit{arg}\texttt{)}

\PEFuncRef{ReplaceWithReference}

\noindent\texttt{ReplaceWithReference(}[\textit{arg}], [\textit{arg}]\texttt{)}

\PEFuncRef{Revert}

\noindent\texttt{Revert(}\texttt{)}

\PEFuncRef{RevertToBackup}

\noindent\texttt{RevertToBackup(}\texttt{)}

\PEFuncRef{Rotate}

\noindent\texttt{Rotate(}\textit{arg}, [\textit{arg}], [\textit{arg}]\texttt{)}

\PEFuncRef{Round}

\noindent\texttt{Round(}\textit{arg}\texttt{)}

Returns the nearest integer to $x$, with ties broken by returning the
nearest \emph{even} integer.  This behaviour may differ on nonstandard
systems that lack the \texttt{fesetround()} library function.
This function may be used without a loaded font.

\PEFuncRef{RoundToCluster}

\noindent\texttt{RoundToCluster(}[\textit{arg}], [\textit{arg}]\texttt{)}

\PEFuncRef{RoundToInt}

\noindent\texttt{RoundToInt(}\textit{arg}\texttt{)}

Round coordinates in selected glyphs.  See
\hyperref[func:Round]{\texttt{Round()}} for rounding a
real number to an integer.

\PEFuncRef{SameGlyphAs}

\noindent\texttt{SameGlyphAs(}\texttt{)}

\PEFuncRef{Save}

\noindent\texttt{Save(}[\textit{arg}], [\textit{arg}]\texttt{)}

\PEFuncRef{SaveTableToFile}

\noindent\texttt{SaveTableToFile(}\textit{arg}, \textit{arg}\texttt{)}

\PEFuncRef{Scale}

\noindent\texttt{Scale(}\textit{arg}, \textit{arg}, [\textit{arg}], [\textit{arg}]\texttt{)}

\PEFuncRef{ScaleToEm}

\noindent\texttt{ScaleToEm(}\textit{arg}, \textit{arg}\texttt{)}

\PEFuncRef{Select}

\noindent\texttt{Select(}, \ldots\texttt{)}

\PEFuncRef{SelectAll}

\noindent\texttt{SelectAll(}\texttt{)}

\PEFuncRef{SelectAllInstancesOf}

\noindent\texttt{SelectAllInstancesOf(}, \ldots\texttt{)}

\PEFuncRef{SelectBitmap}

\noindent\texttt{SelectBitmap(}\textit{arg}\texttt{)}

\PEFuncRef{SelectByATT}

\noindent\texttt{SelectByATT(}, \ldots\texttt{)}

\PEFuncRef{SelectByColor}

\noindent\texttt{SelectByColor(}\textit{arg}\texttt{)}

\PEFuncRef{SelectByColour}

\noindent\texttt{SelectByColour(}\textit{arg}\texttt{)}

\PEFuncRef{SelectByPosSub}

\noindent\texttt{SelectByPosSub(}\textit{arg}, \textit{arg}\texttt{)}

\PEFuncRef{SelectChanged}

\noindent\texttt{SelectChanged(}\textit{arg}\texttt{)}

\PEFuncRef{SelectFewer}

\noindent\texttt{SelectFewer(}\textit{arg}, \ldots\texttt{)}

\PEFuncRef{SelectFewerSingletons}

\noindent\texttt{SelectFewerSingletons(}\textit{arg}, \ldots\texttt{)}

\PEFuncRef{SelectGlyphsBoth}

\noindent\texttt{SelectGlyphsBoth(}\textit{arg}\texttt{)}

\PEFuncRef{SelectGlyphsReferences}

\noindent\texttt{SelectGlyphsReferences(}\textit{arg}\texttt{)}

\PEFuncRef{SelectGlyphsSplines}

\noindent\texttt{SelectGlyphsSplines(}\textit{arg}\texttt{)}

\PEFuncRef{SelectHintingNeeded}

\noindent\texttt{SelectHintingNeeded(}\textit{arg}\texttt{)}

\PEFuncRef{SelectIf}

\noindent\texttt{SelectIf(}, \ldots\texttt{)}

\PEFuncRef{SelectInvert}

\noindent\texttt{SelectInvert(}\texttt{)}

\PEFuncRef{SelectMore}

\noindent\texttt{SelectMore(}\textit{arg}, \ldots\texttt{)}

\PEFuncRef{SelectMoreIf}

\noindent\texttt{SelectMoreIf(}\textit{arg}, \ldots\texttt{)}

\PEFuncRef{SelectMoreSingletons}

\noindent\texttt{SelectMoreSingletons(}\textit{arg}, \ldots\texttt{)}

\PEFuncRef{SelectMoreSingletonsIf}

\noindent\texttt{SelectMoreSingletonsIf(}\textit{arg}, \ldots\texttt{)}

\PEFuncRef{SelectNone}

\noindent\texttt{SelectNone(}\texttt{)}

\PEFuncRef{SelectSingletons}

\noindent\texttt{SelectSingletons(}, \ldots\texttt{)}

\PEFuncRef{SelectSingletonsIf}

\noindent\texttt{SelectSingletonsIf(}, \ldots\texttt{)}

\PEFuncRef{SelectWorthOutputting}

\noindent\texttt{SelectWorthOutputting(}\textit{arg}\texttt{)}

\PEFuncRef{SetCharCnt}

\noindent\texttt{SetCharCnt(}\textit{arg}\texttt{)}

\PEFuncRef{SetCharColor}

\noindent\texttt{SetCharColor(}\textit{arg}\texttt{)}

\PEFuncRef{SetCharComment}

\noindent\texttt{SetCharComment(}\textit{arg}\texttt{)}

\PEFuncRef{SetCharCounterMask}

\noindent\texttt{SetCharCounterMask(}\textit{arg}, \textit{arg}, \ldots\texttt{)}

\PEFuncRef{SetCharName}

\noindent\texttt{SetCharName(}\textit{arg}, \textit{arg}\texttt{)}

\PEFuncRef{SetFeatureList}

\noindent\texttt{SetFeatureList(}\textit{arg}, \textit{arg}\texttt{)}

\PEFuncRef{SetFondName}

\noindent\texttt{SetFondName(}\textit{arg}\texttt{)}

\PEFuncRef{SetFontHasVerticalMetrics}

\noindent\texttt{SetFontHasVerticalMetrics(}\textit{arg}\texttt{)}

\PEFuncRef{SetFontNames}

\noindent\texttt{SetFontNames(}\textit{arg}, \textit{arg}, [\textit{arg}], [\textit{arg}], [\textit{arg}], [\textit{arg}]\texttt{)}

\PEFuncRef{SetFontOrder}

\noindent\texttt{SetFontOrder(}\textit{arg}\texttt{)}

\PEFuncRef{SetGasp}

\noindent\texttt{SetGasp(}, \ldots\texttt{)}

\PEFuncRef{SetGlyphChanged}

\noindent\texttt{SetGlyphChanged(}\textit{arg}\texttt{)}

\PEFuncRef{SetGlyphClass}

\noindent\texttt{SetGlyphClass(}\textit{arg}\texttt{)}

\PEFuncRef{SetGlyphColor}

\noindent\texttt{SetGlyphColor(}\textit{arg}\texttt{)}

\PEFuncRef{SetGlyphComment}

\noindent\texttt{SetGlyphComment(}\textit{arg}\texttt{)}

\PEFuncRef{SetGlyphCounterMask}

\noindent\texttt{SetGlyphCounterMask(}, \ldots\texttt{)}

\PEFuncRef{SetGlyphName}

\noindent\texttt{SetGlyphName(}, \ldots\texttt{)}

\PEFuncRef{SetGlyphTeX}

\noindent\texttt{SetGlyphTeX(}\textit{arg}, \textit{arg}, \textit{arg}, [\textit{arg}]\texttt{)}

\PEFuncRef{SetItalicAngle}

\noindent\texttt{SetItalicAngle(}\textit{arg}, \textit{arg}\texttt{)}

\PEFuncRef{SetKern}

\noindent\texttt{SetKern(}\textit{arg}, \textit{arg}, \textit{arg}\texttt{)}

\PEFuncRef{SetLBearing}

\noindent\texttt{SetLBearing(}\textit{arg}, \textit{arg}\texttt{)}

\PEFuncRef{SetMacStyle}

\noindent\texttt{SetMacStyle(}\textit{arg}\texttt{)}

\PEFuncRef{SetMaxpValue}

\noindent\texttt{SetMaxpValue(}\textit{arg}, \textit{arg}\texttt{)}

\PEFuncRef{SetOS2Value}

\noindent\texttt{SetOS2Value(}, \ldots\texttt{)}

\PEFuncRef{SetPanose}

\noindent\texttt{SetPanose(}\textit{arg}, \textit{arg}\texttt{)}

\PEFuncRef{SetPref}

\noindent\texttt{SetPref(}\textit{arg}, \textit{arg}, \textit{arg}\texttt{)}

This function may be used without a loaded font.

\PEFuncRef{SetRBearing}

\noindent\texttt{SetRBearing(}\textit{arg}, \textit{arg}\texttt{)}

\PEFuncRef{SetTTFName}

\noindent\texttt{SetTTFName(}\textit{arg}, \textit{arg}, \textit{arg}\texttt{)}

\PEFuncRef{SetTeXParams}

\noindent\texttt{SetTeXParams(}, \ldots\texttt{)}

\PEFuncRef{SetUnicodeValue}

\noindent\texttt{SetUnicodeValue(}\textit{arg}, \textit{arg}\texttt{)}

\PEFuncRef{SetUniqueID}

\noindent\texttt{SetUniqueID(}\textit{arg}\texttt{)}

\PEFuncRef{SetVKern}

\noindent\texttt{SetVKern(}\textit{arg}, \textit{arg}, \textit{arg}\texttt{)}

\PEFuncRef{SetVWidth}

\noindent\texttt{SetVWidth(}\textit{arg}, \textit{arg}\texttt{)}

\PEFuncRef{SetWidth}

\noindent\texttt{SetWidth(}\textit{arg}, \textit{arg}\texttt{)}

\PEFuncRef{Shadow}

\noindent\texttt{Shadow(}\textit{arg}, \textit{arg}, \textit{arg}\texttt{)}

\PEFuncRef{Shell}

\noindent\texttt{Shell(}\textit{arg}\texttt{)}

This function may be used without a loaded font.

\PEFuncRef{Simplify}

\noindent\texttt{Simplify(}, \ldots\texttt{)}

\PEFuncRef{Sin}

\noindent\texttt{Sin(}\textit{arg}\texttt{)}

This function may be used without a loaded font.

\PEFuncRef{SizeOf}

\noindent\texttt{SizeOf(}\textit{arg}\texttt{)}

This function may be used without a loaded font.

\PEFuncRef{Skew}

\noindent\texttt{Skew(}\textit{arg}, \textit{arg}, [\textit{arg}], [\textit{arg}]\texttt{)}

\PEFuncRef{SmallCaps}

\noindent\texttt{SmallCaps(}[\textit{arg}], [\textit{arg}], [\textit{arg}], [\textit{arg}]\texttt{)}

\PEFuncRef{Sqrt}

\noindent\texttt{Sqrt(}\textit{arg}\texttt{)}

This function may be used without a loaded font.

\PEFuncRef{StrJoin}

\noindent\texttt{StrJoin(}\textit{arg}, \textit{arg}\texttt{)}

This function may be used without a loaded font.

\PEFuncRef{StrSplit}

\noindent\texttt{StrSplit(}\textit{arg}, \textit{arg}, \textit{arg}\texttt{)}

This function may be used without a loaded font.

\PEFuncRef{Strcasecmp}

\noindent\texttt{Strcasecmp(}\textit{arg}, \textit{arg}\texttt{)}

This function may be used without a loaded font.

\PEFuncRef{Strcasestr}

\noindent\texttt{Strcasestr(}\textit{arg}, \textit{arg}\texttt{)}

This function may be used without a loaded font.

\PEFuncRef{Strftime}

\noindent\texttt{Strftime(}\textit{arg}, \textit{arg}, [\textit{arg}]\texttt{)}

This function may be used without a loaded font.

\PEFuncRef{Strlen}

\noindent\texttt{Strlen(}\textit{arg}\texttt{)}

This function may be used without a loaded font.

\PEFuncRef{Strrstr}

\noindent\texttt{Strrstr(}\textit{arg}, \textit{arg}\texttt{)}

This function may be used without a loaded font.

\PEFuncRef{Strskipint}

\noindent\texttt{Strskipint(}\textit{arg}, \textit{arg}\texttt{)}

This function may be used without a loaded font.

\PEFuncRef{Strstr}

\noindent\texttt{Strstr(}\textit{arg}, \textit{arg}\texttt{)}

This function may be used without a loaded font.

\PEFuncRef{Strsub}

\noindent\texttt{Strsub(}\textit{arg}, \textit{arg}, \textit{arg}\texttt{)}

This function may be used without a loaded font.

\PEFuncRef{Strtod}

\noindent\texttt{Strtod(}\textit{arg}\texttt{)}

This function may be used without a loaded font.

\PEFuncRef{Strtol}

\noindent\texttt{Strtol(}\textit{arg}, \textit{arg}\texttt{)}

This function may be used without a loaded font.

\PEFuncRef{SubstitutionPoints}

\noindent\texttt{SubstitutionPoints(}\texttt{)}

\PEFuncRef{Tan}

\noindent\texttt{Tan(}\textit{arg}\texttt{)}

This function may be used without a loaded font.

\PEFuncRef{ToLower}

\noindent\texttt{ToLower(}\textit{arg}\texttt{)}

This function may be used without a loaded font.

\PEFuncRef{ToMirror}

\noindent\texttt{ToMirror(}\textit{arg}\texttt{)}

This function may be used without a loaded font.

\PEFuncRef{ToString}

\noindent\texttt{ToString(}\textit{arg}\texttt{)}

This function may be used without a loaded font.

\PEFuncRef{ToUpper}

\noindent\texttt{ToUpper(}\textit{arg}\texttt{)}

This function may be used without a loaded font.

\PEFuncRef{Transform}

\noindent\texttt{Transform(}\textit{arg}, \textit{arg}, \textit{arg}, \textit{arg}, \textit{arg}, \textit{arg}\texttt{)}

\PEFuncRef{TypeOf}

\noindent\texttt{TypeOf(}\textit{arg}\texttt{)}

This function may be used without a loaded font.

\PEFuncRef{UCodePoint}

\noindent\texttt{UCodePoint(}\textit{arg}\texttt{)}

This function may be used without a loaded font.

\PEFuncRef{Ucs4}

\noindent\texttt{Ucs4(}\textit{arg}\texttt{)}

This function may be used without a loaded font.

\PEFuncRef{UnicodeAnnotationFromLib}

\noindent\texttt{UnicodeAnnotationFromLib(}\textit{arg}\texttt{)}

This function may be used without a loaded font.

\PEFuncRef{UnicodeBlockEndFromLib}

\noindent\texttt{UnicodeBlockEndFromLib(}\textit{arg}\texttt{)}

This function may be used without a loaded font.

\PEFuncRef{UnicodeBlockNameFromLib}

\noindent\texttt{UnicodeBlockNameFromLib(}\textit{arg}\texttt{)}

This function may be used without a loaded font.

\PEFuncRef{UnicodeBlockStartFromLib}

\noindent\texttt{UnicodeBlockStartFromLib(}\textit{arg}\texttt{)}

This function may be used without a loaded font.

\PEFuncRef{UnicodeFromName}

\noindent\texttt{UnicodeFromName(}\textit{arg}\texttt{)}

This function may be used without a loaded font.

\PEFuncRef{UnicodeNameFromLib}

\noindent\texttt{UnicodeNameFromLib(}\textit{arg}\texttt{)}

This function may be used without a loaded font.

\PEFuncRef{UnicodeNamesListVersion}

\noindent\texttt{UnicodeNamesListVersion(}\texttt{)}

This function may be used without a loaded font.

\PEFuncRef{UnlinkReference}

\noindent\texttt{UnlinkReference(}\texttt{)}

\PEFuncRef{Utf8}

\noindent\texttt{Utf8(}\textit{arg}\texttt{)}

This function may be used without a loaded font.

\PEFuncRef{VFlip}

\noindent\texttt{VFlip(}\textit{arg}\texttt{)}

\PEFuncRef{VKernFromHKern}

\noindent\texttt{VKernFromHKern(}\texttt{)}

\PEFuncRef{Validate}

\noindent\texttt{Validate(}[\textit{arg}]\texttt{)}

\PEFuncRef{Wireframe}

\noindent\texttt{Wireframe(}\textit{arg}, \textit{arg}, \textit{arg}\texttt{)}

\PEFuncRef{WorthOutputting}

\noindent\texttt{WorthOutputting(}\textit{arg}\texttt{)}

\PEFuncRef{WriteStringToFile}

\noindent\texttt{WriteStringToFile(}\textit{arg}, \textit{arg}, \textit{arg}\texttt{)}

This function may be used without a loaded font.

\PEFuncRef{WritePfm}

\noindent\texttt{WritePfm(}\textit{arg}\texttt{)}
%%%%%%%%%%%%%%%%%%%%%%%%%%%%%%%%%%%%%%%%%%%%%%%%%%%%%%%%%%%%%%%%%%%%%%%%

\section{Built-in functions in FontAnvil and not in FontForge}

\PEFuncRef{Shell}

Shell:  takes one argument, the name of a shell command to execute.  Returns
the return value from doing so.

\section{Built-in functions in FontForge and not in FontAnvil}

Because FontAnvil does not support plugins, these built-in functions have
been removed from the language:  LoadPlugin, LoadPluginDir.

Similarly, FontAnvil does not store persistent preferences in the user's
home directory, and the functions to do that have been removed: LoadPrefs,
SavePrefs.  For the moment, at least, other functions related to
``preference'' variables remain in the language.

FontAnvil (in the current version) does not support printing fonts, because
support for this feature necessitates a disproportionate amount of
unportable interfacing code to talk to system-specific printing interfaces. 
Some future version may support a stripped-down and portable printing
feature, likely writing to files instead of the physical printer, but for
the moment, these functions are unimplemented:  PrintFont, PrintSetup.

In FontForge's history it has several times happened that function names
were misspelled, or documented incorrectly.  After the errors were
discovered the names were changed in the documents to reflect the designer's
intentions, but the wrong names were kept in the code as aliases of the
correct ones, in order to avoid breaking any existing scripts that might
have relied on them.  The plan is to remove most if not all of these in
FontAnvil, bringing the code closer to the documentation.  To date, the
function aliases of this kind removed from FontAnvil are: bAutoCounter,
bDontAutoHint, bSubstitutionPoints, BuildComposit, GetPrefs.

In mainline FontForge, some functions were deprecated and had their
implementations replaced by error messages.  In FontAnvil these functions
have been removed entirely:  PrivateToCvt, RemoveATT.

\section{Other notes}

FIXME

